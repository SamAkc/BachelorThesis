% !TeX program = lualatex
\PassOptionsToPackage{english,ngerman}{babel}
\documentclass[
	11pt,
	german,
	singlespacing,
	parskip,
	nohyperref,
	consistentlayout,
]{PsiThesis}

\input{misc/setup.tex}
\input{misc/commands.tex}

\hypersetup{colorlinks=false}

\addbibresource{literature.bib}

% THESIS INFORMATION
\newcommand{\thesistype}{Bachelor}
\thesistitle{Qualitative Analyse von Stakeholdern in einem Unternehmen zu Sichtweisen auf Privatsphäre bei Softwareentwicklern bzw. -administratoren}
\def\tdate{\today}
\def\tdate{2. August 2020}
\author{Samet Murat Akcabay}
\supervisor{Prof. Dr. Dominik Herrmann}
\university{\href{https://www.uni-bamberg.de/en/}{Otto-Friedrich-Universität Bamberg}}
\group{\href{https://www.uni-bamberg.de/informatik/psi/}{Lehrstuhl für Privatsphäre und Sicherheit in Informationssystemen}}
\department{Fakultät für Wirtschaftsinformatik und Angewandte Informatik}
\addresses{Brennerstraße 50\\96052 Bamberg}
\subject{subject not used}
\keywords{keywords not used}
% END OF THESIS INFORMATION

% DOCUMENT
\begin{document}
	\selectlanguage{ngerman}
	\frenchspacing
	\frontmatter
	\hypersetup{urlcolor=black}
	\include{misc/titlepage}
	\hypersetup{urlcolor=ubblue80}

	% ABSTRACT
	\begin{abstract}
	Diese qualitative Analyse beschäftigt sich mit den Sichtweisen auf Privatsphäre und Sicherheit von Stakeholdern in einem Softwareentwicklungs-Unternehmen. 
	Mit der stetig wachsenden Digitalisierung spielt der Datenschutz, und somit die Privatsphäre und Sicherheit, eine immer größere Rolle bei Nutzern des Internets. 
	Aus praktischer Sicht sind die Sichtweisen auf diese unterschiedlich, weswegen bei der Recherche darauf Wert gelegt wurde, möglichst viele Experten aus 
	unterschiedlichen Berufsgruppen (z.B. Werkstudenten, Festangestellte, Team-Leiter o.Ä.) einzubeziehen und deren Aussagen auszuwerten. 
	Nach erfolgreicher Auswertung der Expertenbefragungen hat sich ergeben, dass die Berufsgruppe, damit verbunden die Arbeitsweise und der -umfang, [...]
	keinen Einfluss auf die Einstellung gegenüber personenbezogenen Daten und die Verarbeitung dieser in DevOps-Tools haben.
	\end{abstract}
	% END OF ABSTRACT
	
	% !TeX program = lualatex
\PassOptionsToPackage{english,ngerman}{babel}
\documentclass[
	11pt,
	german,
	singlespacing,
	parskip,
	nohyperref,
	consistentlayout,
]{PsiThesis}

\input{misc/setup.tex}
\input{misc/commands.tex}

\hypersetup{colorlinks=false}

\addbibresource{literature.bib}

% THESIS INFORMATION
\newcommand{\thesistype}{Bachelor}
\thesistitle{Qualitative Analyse von Stakeholdern in einem Unternehmen zu Sichtweisen auf Privatsphäre bei Softwareentwicklern bzw. -administratoren}
\def\tdate{\today}
\def\tdate{2. August 2020}
\author{Samet Murat Akcabay}
\supervisor{Prof. Dr. Dominik Herrmann}
\university{\href{https://www.uni-bamberg.de/en/}{Otto-Friedrich-Universität Bamberg}}
\group{\href{https://www.uni-bamberg.de/informatik/psi/}{Lehrstuhl für Privatsphäre und Sicherheit in Informationssystemen}}
\department{Fakultät für Wirtschaftsinformatik und Angewandte Informatik}
\addresses{Brennerstraße 50\\96052 Bamberg}
\subject{subject not used}
\keywords{keywords not used}
% END OF THESIS INFORMATION

% DOCUMENT
\begin{document}
	\selectlanguage{ngerman}
	\frenchspacing
	\frontmatter
	\hypersetup{urlcolor=black}
	\include{misc/titlepage}
	\hypersetup{urlcolor=ubblue80}

	% ABSTRACT
	\begin{abstract}
	Diese qualitative Analyse beschäftigt sich mit den Sichtweisen auf Privatsphäre und Sicherheit von Stakeholdern in einem Softwareentwicklungs-Unternehmen. 
	Mit der stetig wachsenden Digitalisierung spielt der Datenschutz, und somit die Privatsphäre und Sicherheit, eine immer größere Rolle bei Nutzern des Internets. 
	Aus praktischer Sicht sind die Sichtweisen auf diese unterschiedlich, weswegen bei der Recherche darauf Wert gelegt wurde, möglichst viele Experten aus 
	unterschiedlichen Berufsgruppen (z.B. Werkstudenten, Festangestellte, Team-Leiter o.Ä.) einzubeziehen und deren Aussagen auszuwerten. 
	Nach erfolgreicher Auswertung der Expertenbefragungen hat sich ergeben, dass die Berufsgruppe, damit verbunden die Arbeitsweise und der -umfang, [...]
	keinen Einfluss auf die Einstellung gegenüber personenbezogenen Daten und die Verarbeitung dieser in DevOps-Tools haben.
	\end{abstract}
	% END OF ABSTRACT
	
	% !TeX program = lualatex
\PassOptionsToPackage{english,ngerman}{babel}
\documentclass[
	11pt,
	german,
	singlespacing,
	parskip,
	nohyperref,
	consistentlayout,
]{PsiThesis}

\input{misc/setup.tex}
\input{misc/commands.tex}

\hypersetup{colorlinks=false}

\addbibresource{literature.bib}

% THESIS INFORMATION
\newcommand{\thesistype}{Bachelor}
\thesistitle{Qualitative Analyse von Stakeholdern in einem Unternehmen zu Sichtweisen auf Privatsphäre bei Softwareentwicklern bzw. -administratoren}
\def\tdate{\today}
\def\tdate{2. August 2020}
\author{Samet Murat Akcabay}
\supervisor{Prof. Dr. Dominik Herrmann}
\university{\href{https://www.uni-bamberg.de/en/}{Otto-Friedrich-Universität Bamberg}}
\group{\href{https://www.uni-bamberg.de/informatik/psi/}{Lehrstuhl für Privatsphäre und Sicherheit in Informationssystemen}}
\department{Fakultät für Wirtschaftsinformatik und Angewandte Informatik}
\addresses{Brennerstraße 50\\96052 Bamberg}
\subject{subject not used}
\keywords{keywords not used}
% END OF THESIS INFORMATION

% DOCUMENT
\begin{document}
	\selectlanguage{ngerman}
	\frenchspacing
	\frontmatter
	\hypersetup{urlcolor=black}
	\include{misc/titlepage}
	\hypersetup{urlcolor=ubblue80}

	% ABSTRACT
	\begin{abstract}
	Diese qualitative Analyse beschäftigt sich mit den Sichtweisen auf Privatsphäre und Sicherheit von Stakeholdern in einem Softwareentwicklungs-Unternehmen. 
	Mit der stetig wachsenden Digitalisierung spielt der Datenschutz, und somit die Privatsphäre und Sicherheit, eine immer größere Rolle bei Nutzern des Internets. 
	Aus praktischer Sicht sind die Sichtweisen auf diese unterschiedlich, weswegen bei der Recherche darauf Wert gelegt wurde, möglichst viele Experten aus 
	unterschiedlichen Berufsgruppen (z.B. Werkstudenten, Festangestellte, Team-Leiter o.Ä.) einzubeziehen und deren Aussagen auszuwerten. 
	Nach erfolgreicher Auswertung der Expertenbefragungen hat sich ergeben, dass die Berufsgruppe, damit verbunden die Arbeitsweise und der -umfang, [...]
	keinen Einfluss auf die Einstellung gegenüber personenbezogenen Daten und die Verarbeitung dieser in DevOps-Tools haben.
	\end{abstract}
	% END OF ABSTRACT
	
	% !TeX program = lualatex
\PassOptionsToPackage{english,ngerman}{babel}
\documentclass[
	11pt,
	german,
	singlespacing,
	parskip,
	nohyperref,
	consistentlayout,
]{PsiThesis}

\input{misc/setup.tex}
\input{misc/commands.tex}

\hypersetup{colorlinks=false}

\addbibresource{literature.bib}

% THESIS INFORMATION
\newcommand{\thesistype}{Bachelor}
\thesistitle{Qualitative Analyse von Stakeholdern in einem Unternehmen zu Sichtweisen auf Privatsphäre bei Softwareentwicklern bzw. -administratoren}
\def\tdate{\today}
\def\tdate{2. August 2020}
\author{Samet Murat Akcabay}
\supervisor{Prof. Dr. Dominik Herrmann}
\university{\href{https://www.uni-bamberg.de/en/}{Otto-Friedrich-Universität Bamberg}}
\group{\href{https://www.uni-bamberg.de/informatik/psi/}{Lehrstuhl für Privatsphäre und Sicherheit in Informationssystemen}}
\department{Fakultät für Wirtschaftsinformatik und Angewandte Informatik}
\addresses{Brennerstraße 50\\96052 Bamberg}
\subject{subject not used}
\keywords{keywords not used}
% END OF THESIS INFORMATION

% DOCUMENT
\begin{document}
	\selectlanguage{ngerman}
	\frenchspacing
	\frontmatter
	\hypersetup{urlcolor=black}
	\include{misc/titlepage}
	\hypersetup{urlcolor=ubblue80}

	% ABSTRACT
	\begin{abstract}
	Diese qualitative Analyse beschäftigt sich mit den Sichtweisen auf Privatsphäre und Sicherheit von Stakeholdern in einem Softwareentwicklungs-Unternehmen. 
	Mit der stetig wachsenden Digitalisierung spielt der Datenschutz, und somit die Privatsphäre und Sicherheit, eine immer größere Rolle bei Nutzern des Internets. 
	Aus praktischer Sicht sind die Sichtweisen auf diese unterschiedlich, weswegen bei der Recherche darauf Wert gelegt wurde, möglichst viele Experten aus 
	unterschiedlichen Berufsgruppen (z.B. Werkstudenten, Festangestellte, Team-Leiter o.Ä.) einzubeziehen und deren Aussagen auszuwerten. 
	Nach erfolgreicher Auswertung der Expertenbefragungen hat sich ergeben, dass die Berufsgruppe, damit verbunden die Arbeitsweise und der -umfang, [...]
	keinen Einfluss auf die Einstellung gegenüber personenbezogenen Daten und die Verarbeitung dieser in DevOps-Tools haben.
	\end{abstract}
	% END OF ABSTRACT
	
	\include{main.tex}

	% TABLE OF CONTENTS
	\cleardoublepage

	\newgeometry{
		head=13.6pt,
		top=27.4mm,
		bottom=27.4mm,
		inner=24.8mm,
		outer=24.8mm,
		marginparsep=0mm,
		marginparwidth=0mm,
	}
	{
		\hypersetup{linkcolor=black}
		\tableofcontents % Prints the ToC entries
	}
	\restoregeometry

	% THESIS CONTENTS - CHAPTERS
	\mainmatter
	\pagestyle{thesis}

	\newcommand{\keyword}[1]{\textbf{#1}}
	\newcommand{\tabhead}[1]{\textbf{#1}}
	\newcommand{\code}[1]{\texttt{#1}}
	\newcommand{\file}[1]{\texttt{#1}}
	\newcommand{\option}[1]{\texttt{\itshape#1}}

	\graphicspath{{./figures/}{./examples/}}

	% CHAPTERS
	\input{chapters/introduction.tex}
	\input{chapters/background.tex}
	\input{chapters/research.tex}
	\input{chapters/results.tex}
	\input{chapters/discussion.tex}
	\input{chapters/conclusion.tex}

	% APPENDICES
	\addtocontents{toc}{\string\def\string\chaptername{Appendix}}
	\appendix
	\renewcommand{\thesection}{\thechapter.\arabic{section}}
	\renewcommand{\thesubsection}{\thesection.\arabic{subsection}}
	\renewcommand{\thesubsubsection}{\thesubsection.\arabic{subsubsection}}

	\include{appendices/appendixA}
	\include{appendices/appendixB}
	%\include{appendices/appendixC}

	% BIBLIOGRAPHY
	\newgeometry{
		inner=2cm,
		outer=2cm, 
		marginparwidth=0cm,
		marginparsep=0mm,
		bindingoffset=.5cm,
		top=1.5cm,
		bottom=2.5cm,
		includehead,
		includefoot
	}
	\addchap{Literaturverzeichnis}

	\setlength\columnsep{2em}
	\begin{multicols}{2}
		\begin{refcontext}[sorting=nyt]
			\renewcommand*{\bibfont}{\small\RaggedRight}
			\linespread{1.0}\selectfont
			\printbibliography[heading=none]
		\end{refcontext}
	\end{multicols}

	% DECLARATION
	\begin{declaration}
		\addchaptertocentry{\authorshipname}
		Ich erkläre hiermit gemä\ss\ \S~17 Abs.\,2 APO, dass ich die vorstehende {\thesistype}arbeit selbständig\\ verfasst und keine anderen als die angegebenen Quellen und Hilfsmittel benutzt habe.\\
		\bigskip
		\bigskip
		\\
		\begin{tabular}{@{}l@{}}
  		Bamberg, den \rule[-0.8em]{10em}{0.5pt}\\[2ex]
  		~
		\end{tabular}
		\hspace{\fill}%
		\begin{tabular}{@{}c@{}}
  			\rule[-0.8em]{20em}{0.5pt}\\[2ex]
  			\authorname
			\end{tabular}\hspace{\fill}
	\end{declaration}
\end{document}

	% TABLE OF CONTENTS
	\cleardoublepage

	\newgeometry{
		head=13.6pt,
		top=27.4mm,
		bottom=27.4mm,
		inner=24.8mm,
		outer=24.8mm,
		marginparsep=0mm,
		marginparwidth=0mm,
	}
	{
		\hypersetup{linkcolor=black}
		\tableofcontents % Prints the ToC entries
	}
	\restoregeometry

	% THESIS CONTENTS - CHAPTERS
	\mainmatter
	\pagestyle{thesis}

	\newcommand{\keyword}[1]{\textbf{#1}}
	\newcommand{\tabhead}[1]{\textbf{#1}}
	\newcommand{\code}[1]{\texttt{#1}}
	\newcommand{\file}[1]{\texttt{#1}}
	\newcommand{\option}[1]{\texttt{\itshape#1}}

	\graphicspath{{./figures/}{./examples/}}

	% CHAPTERS
	% !TeX program = lualatex
\chapter{Die Relevanz der Privatsphäre im 21. Jahrhundert - eine Einleitung} % Main chapter title
\label{Introduction} % For referencing the chapter elsewhere, use \ref{Introduction}

\enquote{We believe privacy is a fundamental human right} \cite{WWDC:20} - mit dieser Aussage betritt Craig Federighi, der Vizepräsident der Softwareentwicklung von Apple, einem Multimilliarden-Unternehmen der Software- und Hardwareentwicklung aus den Vereinigten Staaten, zur Vorstellung der 
neuen Privatsphäre-Richtliniendes Unternehmens zur sogenannten, firmeneigenen \enquote{Worldwide Developer Conference} die Bühne. \\ Und dieser Glaube ist nicht unbegründet: Mit der zunehmenden Digitalisierung in den letzten Jahrzehnten werden Menschen, und Internetnutzer im Spezifischen, 
häufiger vor dieses Dilemma gestellt. Entwicklungen wie dem Internet der Dinge (IoT), Industrie 4.0, \enquote{Smart Home}, Wearables (diverse Sensoren an der Kleidung selbst, Smartwatches, Fitnesstracker) uvm. ermöglichen neue (und zum Teil unerforschte) Angriffsmöglichkeiten für Cyberkriminalität \cite{BLB:18}.
Dies zeichnet sich vor allem durch den Anstieg der Höhe der Fallzahlen und der gleichzeitigen Abnahme der Aufklärungsquote aus: Im ersten Halbjahr 2018 wurden im Durchschnitt 13.000 Malware-Samples am Tag neu entdeckt \cite{GDB:18}, während diese Zahl 2017 noch bei [...] lag. \\ Ähnlich sieht es hier bei
dem Anstieg neuer Schadprogramme aus: Der Blog von G Data, einem deutschen Softwareunternehmen, welcher mehrfach für seine Sicherheitslösungen ausgezeichnet wurde \cite{GD+1}, beschreibt dabei, dass die Zahl der neuen Schadprogrammtypen seit 2007 einen Anstieg um das knapp 63-fache vermerken konnte \cite{GDB:17}.
	% !TeX program = lualatex
\chapter{Theoretischer Hintergrund} % Main chapter title
\label{Background} % For referencing the chapter elsewhere, use \ref{Background}

\section{Definition und Identifikation der Stakeholder in einem Softwareentwicklungsunternehmen}
Ziel dieser qualitativen Analyse ist es, die Denk- bzw. Sichtweisen von Stakeholdern in Softwareentwicklungsunternehmen zu erfassen und anhand dieser Ergebnisse, Schlüsse in Bezug auf die
eigene Privatsphäre dieser zu ziehen. Um ein grundlegendes Verständnis in den folgenden Kapiteln gewährleisten zu können ist es zunächst notwendig, diese Stakeholder zu definieren und anhand davon, 
potenzielle Stakeholder im Rahmen dieser Forschung zu identifizieren. \\
In den folgenden Unterkapiteln werden die untergliederten Interessensgruppen und ihre entsprechende Notwendigkeit näher erläutert. Die Begründung für die spezifische Wahl dieser besagten Gruppen wird im
nächsten Kapitel klassifiziert.


\subsection{Berufseinsteiger, Praktikanten und Werkstudenten}

\subsection{Softwareentwickler}

\subsection{Team-Leads, Projektleiter (und Geschäftsführer?)}

\subsection{Externe Ansprechpartner}

\section{DevOps und deren Einsatz in Softwareentwicklungsunternehmen}

\subsection{Interne Kommunikationstools}

\subsection{Projektmanagement- und Wiki-Software}

\subsection{Git}

\section{Aktueller Stand und Technologien?}
	% !TeX program = lualatex
\chapter{Forschungsmethode} % Main chapter title
\label{Research} % For referencing the chapter elsewhere, use \ref{Research}

Dabei wurde die Vorgehensweise zur Durchführung von Experteninterviews gewählt: Hierzu wurden einzelne Softwareentwickler von diversen Unternehmen sorgfältig
anhand ihrer Qualifikation, Berufserfahrung und der Position im Unternehmen gewählt - um ein möglichst weites Spektrum an unterschiedlichen Personengruppen abzudecken, wurden diese in Berufseinsteiger,
welche kaum bis relativ wenig Berufserfahrung vorweisen können, mehrjährig Festangestellte Softwareentwickler und Team-Leads bzw. Projektleiter, welche zusätzlich noch verantwortlich für Entwicklergruppen sind,
untergliedert.

\section{Definition und Identifikation der Stakeholder in einem Softwareentwicklungsunternehmen}
Ziel dieser qualitativen Analyse ist es, die Denk- bzw. Sichtweisen von Stakeholdern in Softwareentwicklungsunternehmen zu erfassen und anhand dieser Ergebnisse, Schlüsse in Bezug auf die
eigene Privatsphäre dieser zu ziehen. Um ein grundlegendes Verständnis in den folgenden Kapiteln gewährleisten zu können ist es zunächst notwendig, diese Stakeholder zu definieren und anhand davon, 
potenzielle Stakeholder im Rahmen dieser Forschung zu identifizieren. \\ Als Stakeholder werden jene \enquote{Personen, Gruppen oder Institutionen bezeichnet, die von den Aktivitäten eines Unternehmens 
direkt oder indirekt betroffen sind oder [...] ein Interesse an diesen [...] haben} \cite{Fle:16} - dies kann von Kunden und Lieferanten bis zu den eigenen Mitarbeitern und Eigentümern reichen \cite{Fle:16}. 
Im Falle dieser Analyse wird die Sichtweise der einzelnen Stakeholder auf die eigene Privatsphäre im angestellten Unternehmen betrachtet. \\ Gemessen wird dieser Aspekt durch die Relevanz, wie viel Wert die befragten
Softwareentwickler beispielsweise darauf legen, personenbezogene Daten in einem möglichst weiten Spektrum für einen möglichst langen Zeitraum einsehen zu können und auf der Kehrseite, wie weit sie bereit sind, eigene
personenbezogene Daten preiszugeben bzw. mit Kollegen und anderen Stakeholdern zu teilen, um ein gut funktionierendes Glied eines Teams oder eines Unternehmens sein zu können.

In den folgenden Unterkapiteln werden die untergliederten Interessensgruppen und ihre entsprechende Notwendigkeit näher erläutert. Die Begründung für die spezifische Wahl dieser besagten Gruppen wird im
nächsten Kapitel klassifiziert.

\subsection{Berufseinsteiger, Praktikanten und Werkstudenten}
Die primäre Gruppe zeichnet sich durch Softwareentwickler aus, welche kaum bis wenig Berufserfahrung besitzen und eventuell geringere Qualifikationen, als die anderen Interessensgruppen aufweisen könnten.
Diese Merkmale finden sich in Berufseinsteigern wieder, welche vor Kurzem aus einer abgeschlossenen Ausbildung oder eines abgeschlossenen Studiums stammen und nun im Beruf Softwareentwickler sind. Selbe Punkte
lassen sich allerdings auch auf (längerfristige) Praktikanten und Werkstudenten, welche neben ihrem Studium in einem Unternehmen bis zu 20 Stunden in der Woche tätig sind, übertragen - es ist also wenig Berufserfahrung
vorhanden, aber grundlegende Kenntnisse in der Informatik, vor allem in Bezug auf DevOps, welche im nächsten Unterkapitel näher erläutert werden, lassen sich in diesen Interessensgruppen wiederfinden.

\subsection{Softwareentwickler}

\subsection{Team-Leads, Projektleiter (und Geschäftsführer?)}

\subsection{Externe Ansprechpartner}

\section{DevOps und deren Einsatz in Softwareentwicklungsunternehmen}

\subsection{Interne Kommunikationstools}

\subsection{Projektmanagement- und Wiki-Software}

\subsection{Git}

\section{Aktueller Stand und Technologien?}
	% !TeX program = lualatex
\chapter{Ergebnisse} % Main chapter title
\label{Results} % For referencing the chapter elsewhere, use \ref{Results}

Anhand der Expertenbefragungen konnten insgesamt 18 Kategorien zu den verschiedenen Sichtweisen herausausgearbeitet werden. Hierbei wird zwischen
der Präferenz der Privatsphäre im Allgemeinen oder der Benutzerfreundlichkeit bzw. Einfachheit unterschieden: Kategorien, in welchen Experten überwiegend
(>50\% der Experten sind der selben Ansicht) angegeben haben, die Privatsphäre sei ihnen wichtiger, werdem dem Kapitel \ref{privacy} zugeordnet, wohingegen die Kategorien, in welchen die Privatsphäre
aufgrund der Benutzerfreundlichkeit und Einfachheit vernachlässigt werden konnte, dem Kapitel \ref{noprivacy} zugerechnet werden. Im Anschluss werden die Ergebnisse,
in welchen kein einstimmiges Ergebnis (bei sieben Befragten ergibt dies: Weniger als 28\% der Befragten haben die selbe Meinung) erzielt werden konnte, in Kapitel \ref{noclearresult} thematisiert.

\section{Kategorien mit einer Präferenz für die Privatsphäre} \label{privacy}
\textbf{Interesse an persönlichen Daten in Log-Dateien:} \newline
Dieser Punkt befasst sich mit dem allgemeinen Interesse an persönlichen Daten, welche in Form von Benutzernamen, Zeitstempeln, IDs etc. in Log-Dateien sämtlicher Art auftreten können. Hier haben die Befragten einstimmig
entschieden, kein Interesse an diesen zu haben und ein mögliches Entfernen in der Zukunft willkommen zu heißen. Es wurde lediglich angegeben, dass in bestimmten Fällen (Zusammenarbeit an einem Projekt, 
in Testfällen von Code, Servern oder Applikationen oder auf Anfrage) eine Nachverfolgung auf Wunsch eines Kunden oder eines Auftrags erfolgen muss. Für die Befragten ist es nur wichtig, den Ablauf und auftretende Fehler 
von Servern, Applikationen oder Code im Allgemeinen nachverfolgen zu können - ein Interesse an bestimmten Personen oder Zeitstempeln ist nicht vorhanden, weswegen in dieser Kategorie die Privatsphäre über der Nutzerfreundlichkeit
steht. \newline \newline
\textbf{Bevorzugte Aufbewahrungsdauer von Log-Dateien:} \newline
Als Weiterleitung der Frage über die tatsächliche Aufbewahrungsdauer von Log-Dateien in Kapitel \ref{noprivacy} wurden in dieser Kategorie die Experten dazu angehalten, ihre persönliche Präferenz zur Dauer anzugeben. Hier haben alle 
Befragten angegeben, eine temporäre Aufbewahrungsdauer zu bevorzugen. Diese könne anhand der Lebensdauer von Projekten, der Lebensdauer von Programmcode, den letzten x Builds, einer festgelegten Zeit (z.B. zwei Wochen)
oder der Existenz von gemeldeten Fehlern deklariert werden. In der Regel sind sich die Befragten einig, eine Dauer von maximal wenigen Monaten zu bevorzugen. Diese Sichtweisen stellen einen Gegensatz zur tatsächlichen Aufbewahrungsdauer von
Log-Dateien in Softwareentwicklungsunternehmen dar, welche in Kapitel \ref{noprivacy} angesprochen werden.

\section{Kategorien mit einer Präferenz für die Benutzerfreundlichkeit und Einfachheit} \label{noprivacy}
\textbf{Verarbeitete Daten zur Unterstützung der Softwareentwicklung in DevOps-Tools:} \newline
Zu Beginn der Interviews haben alle Befragten angegeben, mit persönlichen Daten in Berührung zu kommen und diese auch zu verarbeiten. Dabei haben alle vier Softwareentwickler angegeben, die Nutzeraktivitäten und den Programmcode von Kollegen 
einsehen und verarbeiten zu können. Dies erfolge zur Nachvollziehbarkeit der Builds und Deployments, des Programmcodes selbst, z.B. durch Git Commits und zur allgemeinen Fehlerbehebung bei fehlgeschlagenen Builds bzw. fehlerhaftem Programmcode. 
Zwei Befragte haben zudem angegeben, Zugriff auf sensible Kundendaten in Form von Klarnamen, Anschriften, E-Mail Adressen uvm. zu besitzen: Dies ist bei jenen der Fall, die häufig mit externen Ansprechpartnern und Unternehmen in Kontakt treten 
und mit diesen zusammenarbeiten müssen. \newline \newline
\textbf{Tatsächliche Aufbewahrungsdauer von Log-Dateien:} \newline
Da ein sehr großer Anteil der Befragten von über 85\% (vgl. Tabelle \ref{tab:generaldata}) angegeben haben, Log-Dateien aktiv zu nutzen, ist es für den weiteren Verlauf der Analyse erforderlich, die tatsächliche Aufbewahrungsdauer der genutzten Log-Dateien 
im Unternehmen zu spezifizieren. Hier haben alle Befragten angegeben, dass alle Log-Dateien in der Regel für eine unbegrenzte bzw. unbestimmte Dauer aufbewahrt werden. Die Stakeholder haben zudem meist keine Kenntnis über die genaue Aufbewahrungsdauer und 
beziehen sich bei ihrer Aussage auf die Existenz von sehr alten Log-Dateien (über 5 Jahre). Die Ausnahme dieser Regel betrifft manuell durchgeführte Bereinigungen von Speicher und eingeleitete Löschung spezifischer Log-Dateien: Bei einem Befragten kam es vor,
dass das System einen Absturz erlitten hat, welche die Log-Dateien unbenutzbar gemacht haben, wohingegen bei einem anderen der Serverspeicher geleert werden musste und dieses Vorgehen sämtliche Log-Dateien gelöscht hat. \newline Aus diesen Aussagen lässt sich 
ableiten, dass ein Vorgehen zur automatisierten Löschung von Log-Dateien bei den Befragten bisher keine Anwendung gefunden hat. \newline \newline
\textbf{Interesse an anderen persönlichen Daten in DevOps-Tools:} \\
\textbf{Gründe für das Interesse an Zeitstempeln in DevOps-Tools:} \\
\textbf{Interesse an persönlichen Daten in Review-Tools:} \\
\textbf{Interesse an persönlichen Daten in Administrationstools:} \\
\textbf{Detailgrad von Issues in Ticketverwaltungstools:} \\
\textbf{Personenbezug von Issue-Kommentaren in Ticketverwaltungstools:} \\
\textbf{Allgemeine Auswertungen von persönlichen Daten aus DevOps-Tools:} \\
\textbf{Relevanz von persönlichen Daten in Builds:} \\
\textbf{Interesse an Log-Dateien:} \\
\textbf{Mögliche Verarbeitungsinteressen von persönlichen Daten in anderen Unternehmen:} \\

\section{Kategorien mit unstimmigen Ergebnissen} \label{noclearresult}
\textbf{Detailgrad von Zeitstempeln in DevOps-Tools:} \\
\textbf{Sonstige Verarbeitungsinteressen von persönlichen Daten:} \\

\section{Vorgeschlagene Alternativen}

\section{Zusammenstellung der Ergebnisse}





	% !TeX program = lualatex
\chapter{Diskussion der Ergebnisse} % Main chapter title
\label{Discussion} % For referencing the chapter elsewhere, use \ref{Discussion}

\section{Implikationen in der Theorie}

\section{Implikationen in der Praxis}

\section{Limitation der Arbeit}
	% !TeX program = lualatex
\chapter{Fazit und Ausblick} % Main chapter title
\label{Conclusion} % For referencing the chapter elsewhere, use \ref{Conclusion}

Im Umfang dieser Analyse wurde die Privatsphäre in Verbindung mit Stakeholdern und DevOps in Softwareentwicklungsunternehmen präsentiert
und anhand Belege aktueller Forschungen ein erster Versuch einer praktischen Feldanalyse gewagt. Anhand der Expertenbefragungen konnten so einzelne
Kategorisierungen herausgearbeitet werden, welche entweder einer Präferenz zur Einfachheit und Benutzerfreundlichkeit oder der Privatsphäre im Arbeitsalltag
und -ablauf zugeordnet wurden. Die einzelnen Kategorien deckten sich dabei mit den Annahmen aus der Forschung. \newline
Da eine so geringe Anzahl an gewählten Experten für eine Analyse jedoch nicht als vollständig repräsentativ angesehen werden kann, ist diese Arbeit als eine
Unterstützung der aktuellen Forschung in Bezug auf die Privatsphäre der Stakeholder und deren Sichtweisen am Arbeitsplatz in Softwareentwicklungsunternehmen
zu betrachten. \newline \newline
Nun bleibt zu hoffen, dass zusätzlich zu den aktuellen Entwicklungen, welche mit der zunehmenden Digitalisierung Eintritt in den Alltag aller finden, neue Entdeckungen
auch in Hinsicht auf Privatsphäre in der Zukunft kritisch begutachtet werden. Hilfreich ist es allemal, wenn \enquote{Big Player} der Software- bzw. Hardwareentwicklung
ein Vorbild repräsentieren und die Relevanz der Privatsphäre immer wieder betonen. \newline Vielleicht überlegt es sich manch einer auf diese Weise zweimal, eine Kundenkarte
wie Payback, der DeutschlandCard o.Ä. zu beantragen und beim Einkaufen zu verwenden.

	% APPENDICES
	\addtocontents{toc}{\string\def\string\chaptername{Appendix}}
	\appendix
	\renewcommand{\thesection}{\thechapter.\arabic{section}}
	\renewcommand{\thesubsection}{\thesection.\arabic{subsection}}
	\renewcommand{\thesubsubsection}{\thesubsection.\arabic{subsubsection}}

	% Appendix A
\chapter{Verbundene Arbeiten}

\section{Stellenausschreibung zur Befragung von Softwareentwicklern bzw. -administratoren} \label{ausschreibung}
Im Rahmen einer Abschlussarbeit, welche sich mit der qualitativen Analyse von Stakeholdern in einem IT-Unternehmen zu Sichtweisen auf Privatsphäre befasst, sind wir auf der Suche nach potenziellen Befragungen von Softwareentwicklern/-administratoren. Sie sollten bereits
\begin{itemize}
    \item in einem Softwareentwicklungsunternehmen gearbeitet haben
    \item Erfahrungen mit der Arbeit im Team haben
    \item mit einem Versionsverwaltungs-Tool (z.B. Git, SVN etc.) gearbeitet haben
    \item grobe Kenntnisse über interne Abläufe besitzen
\end{itemize}
Melden Sie sich bei Interesse bitte bei mir, \textbf{Samet Akcabay}, unter der E-Mail-Adresse \textbf{samet-murat.akcabay@stud.uni-bamberg.de}!

\section{Datenschutzerklärung} \label{erklaerung}
\textbf{Datenschutzerklärung zur Befragung im Rahmen der Abschlussarbeit \enquote{Qualitative Analyse von Stakeholdern in einem Softwareentwicklungsunternehmen zu Sichtweisen auf Privatsphäre bei Softwareentwicklern bzw. -administratoren}} \newline \newline
Wir, der Lehrstuhl für Privatsphäre und Sicherheit in Informationssystemen, legen großen Wert auf Datenschutz und das Recht auf informationelle Selbstbestimmung. Aufgrund dessen möchten wir Sie mit dieser Datenschutzerklärung über die Erhebung, Speicherung und Verarbeitung der Daten, die im Zusammenhang mit der Abschlussarbeit „Qualitative Analyse von Stakeholdern in einem Softwareentwicklungsunternehmen zu Sichtweisen auf Privatsphäre bei Softwareentwicklern bzw. -administratoren“ erhoben werden, informieren. \newline \newline
Alle personenbezogenen Daten werden in keinster Weise betrachtet, verarbeitet oder an eine andere Instanz weitergeleitet. Bei der Befragung geht es lediglich darum, die Interessen in verschiedenen Unternehmen sowie von verschiedenen Interessensgruppen zu analysieren. Jede einzelne Angabe in jeder Befragung wird im Anschluss zu allgemeinen Punkten kategorisiert und zu Mustern zusammengefasst. Nach erfolgreicher Verarbeitung der nicht-personenbezogenen Daten werden sämtliche Aufnahmen und Transkriptionen unwiderruflich gelöscht. \newline \newline
Die befragten Personen willigen bei Teilnahme an der Befragung im Rahmen der Abschlussarbeit dieser Datenschutzerklärung zu. Sie haben jederzeit das Recht, Ihre Angaben zu widerrufen. In diesem Fall werden alle von Ihnen gespeicherten Daten unwiderruflich gelöscht und eine Verarbeitung dieser findet nicht statt. \newline \newline
Sie können jederzeit den Lehrstuhl für Privatsphäre und Sicherheit in Informationssystemen oder den Verfasser der Abschlussarbeit bei Fragen oder Anregungen kontaktieren: \newline \newline
\textbf{Universität Bamberg\newline
Lehrstuhl für Privatsphäre und Sicherheit in Informationssystemen An der Weberei 5\newline
96047 Bamberg\newline \newline
Raum:} WE5/05.064\newline
\textbf{Sekretariat:} WE5/05.063\newline 
\textbf{Telefon:} +49 (0)951–863 2661\newline \newline
\textbf{Samet Akcabay\newline
An der Spinnerei 11 Zimmer-Nr. D29 \newline96047 Bamberg\newline \newline
Telefon:} +49 (0)172-315 5296 \newline
\textbf{E-Mail:} samet-murat.akcabay@stud.uni-bamberg.de


\section{Online-Formular zum Ausfüllen über Microsoft Forms} \label{forms}
\begin{enumerate}
    \item Welche Rolle übernehmen Sie im Unternehmen?
    \item Administrieren Sie auch Server?
    \item Welche Daten verarbeiten Sie allgemein bei der Softwareentwicklung? Welche Daten schauen Sie sich explizit mithilfe von DevOps-Tools an, die Sie bei der Softwareentwicklung unterstützen?
    \item Wo liegen diese Daten, die von Ihnen verarbeitet werden? Finden sich unter diesen auch personenbezogene Daten?
    \item Was bedeutet für Sie allgemein "Daten verarbeiten"? Welche Abläufe bringen Sie damit in Verbindung?
    \item Wie genau können Sie diese von Ihnen genannten Daten einsehen?
    \item Schauen Sie sich bestimmte Daten explizit an bzw. legen Sie Wert darauf, dass bestimmte Daten gegeben sind?
    \item Schauen Sie sich auch explizit an, wer an welchen Issues/Builds (falls vorhanden) in welchem Projekt gearbeitet hat?
    \item Reviewen Sie auch Code von anderen Entwicklern? Falls ja, was genau schauen Sie sich beim verwendeten Review-Tool an (z.B. Name des Reviewers, Zeitstempel, ID etc.)? Worauf legen Sie besonders wert?
    \item Allgemein gefragt: Was wäre für Sie eine angemessene Zeit zur Aufbewahrung von personenbezogenen Daten? Bitte begründen Sie Ihre Antwort.
    \item Wie lange werden die von Ihnen genannten Daten bei Ihnen im Unternehmen tatsächlich aufbewahrt?
    \item Zurück zu Issues/Builds: Schauen Sie sich auch an, wann/ob/von wem der Issue erstellt/bearbeitet/kommentiert wurde?
    \item Welche Interessen verfolgen Sie generell, wenn Sie sich Zeitstempel in DevOps-Tools ansehen? Worauf legen Sie besonders wert?
    \item Von der anderen Seite aus betrachtet: Worauf könnten Sie verzichten?
    \item Was glauben Sie, zu welchem Zweck man diese Tools verwendet? (z.B. Nachvollziehbarkeit, Leistungsbeurteilung, Qualitätssicherung etc.)
    \item Könnten Sie sich andere Auswertungen aus den Daten vorstellen? (Auch, was Sie persönlich noch nicht selbst erlebt/gemacht haben, sondern welche im Interesse des Unternehmens liegen)
    \item Verwenden/arbeiten Sie mit Log-Dateien?
    \item Falls ja: Finden Sie es wichtig, dass Benutzername und Zeitstempel in den Log-Dateien vorhanden sind? Falls nicht, worauf würden Sie stattdessen setzen?
    \item Wie lange werden Log-Dateien bei Ihnen aufbewahrt? (z.B. von Tickets, Builds, Issues etc.)
    \item Fällt Ihnen noch etwas bisher nicht genanntes ein, worauf Sie wert legen und was Sie sich ansehen?
    \item Fallen Ihnen andere Verarbeitungsinteressen bzw. Auswertungsansätze ein, die andere Firmen haben könnten?

\end{enumerate}
	% Appendix B

\chapter{Interviewleitfaden} \label{interview}

% \begin{table*}[t]
%     \caption{Verwendeter Leitfaden für die Befragung zu den Sichtweisen auf die Privatsphäre in Softwareentwicklungsunternehmen (angelehnt an \cite{Helfferich:2011aa})}
%     \label{tab:leitfaden}
%     \centering
%     \small % use smaller fontsize in the table
%     {\renewcommand{\arraystretch}{2} % increase vertical space between rows
%     \begin{tabularx}{\linewidth}{@{}llllX@{}} % @{} omits outer horizontal margins, the "X" column uses up all remaining available space 
%       \toprule
%       Leitfrage & Check & Konkrete Fragen & Aufrechterhaltungs-/Steuerungsfragen \\
%       \midrule
%         \enquote{Welche Daten verarbeiten Sie?}                                                                             & Desinteresse                  & Persönliche Daten in DevOps-Tools             \\
%         \enquote{Was schauen Sie sich konkret an?}                                                                          & Nutzeraktivität               & Persönliche Daten in Administrationstools     \\
%         \enquote{In welcher Auflösung brauchen Sie den Zeitstempel?}                                                        & Desinteresse                  & Nutzung von Tools zur Administration          \\
%         \enquote{Gibt es etwas anderes, was Sie sich ansehen?}                                                              & Desinteresse                  & Persönliche Daten in DevOps-Tools             \\
%         \enquote{Schauen Sie sich auch an, wer spezifisch was gebaut hat?}                                                  & Benutzername                  & Persönliche Daten in DevOps-Tools             \\
%         \enquote{Was schauen Sie sich beim Review-Tool an?}                                                                 & Benutzername                  & Persönliche Daten in DevOps-Tools             \\
%         \enquote{Wie lange ist es für Sie wichtig, dass diese Daten aufbewahrt werden?}                                     & Benutzername                  & Persönliche Daten in DevOps-Tools             \\
%         \enquote{Wie lange werden diese tatsächlich aufbewahrt?}                                                            & Benutzername                  & Persönliche Daten in DevOps-Tools             \\
%         \enquote{Was für Daten schauen Sie sich in Issues an?}                                                              & Benutzername                  & Persönliche Daten in DevOps-Tools             \\
%         \enquote{Schauen Sie sich auch an, von wem ein Kommentar editiert wurde?}                                           & Benutzername                  & Persönliche Daten in DevOps-Tools             \\
%         \enquote{Welche Interessen verfolgen Sie generell bei Zeitstempeln in DevOps-Tools?}                                & Benutzername                  & Persönliche Daten in DevOps-Tools             \\
%         \enquote{Welche Auswertungen, die sie selbst nicht machen, können sie sich sonst mit diesen Daten vorstellen?}      & Benutzername                  & Persönliche Daten in DevOps-Tools             \\
%         \enquote{Schauen Sie sich auch Logs an?}                                                                            & Benutzername                  & Persönliche Daten in DevOps-Tools             \\
%         \enquote{Welche Auswertungen, die sie selbst nicht machen, können sie sich sonst mit diesen Daten vorstellen?}      & Benutzername                  & Persönliche Daten in DevOps-Tools             \\
%         \enquote{Ist es für Sie auch wichtig, wer wann was gemacht hat?}                                                    & Benutzername                  & Persönliche Daten in DevOps-Tools             \\
%         \enquote{Wissen Sie, wie lange diese Log-Dateien aufbewahrt werden?}                                                & Benutzername                  & Persönliche Daten in DevOps-Tools             \\
%         \enquote{Welche Tools verwenden Sie zur Administration?}                                                            & Benutzername                  & Persönliche Daten in DevOps-Tools             \\
%         \enquote{Haben Sie da die Möglichkeit einzusehen, wer wann was gemacht hat?}                                        & Benutzername                  & Persönliche Daten in DevOps-Tools             \\
%         \enquote{Warum ist das \enquote{wer} für Sie relevant?}                                                             & Benutzername                  & Persönliche Daten in DevOps-Tools             \\
%         \enquote{Können Sie sich andere Verarbeitungsinteressen vorstellen?}                                                & Benutzername                  & Persönliche Daten in DevOps-Tools             \\
%       \bottomrule
%     \end{tabularx}
%     }
%   \end{table*}
	%\include{appendices/appendixC}

	% BIBLIOGRAPHY
	\newgeometry{
		inner=2cm,
		outer=2cm, 
		marginparwidth=0cm,
		marginparsep=0mm,
		bindingoffset=.5cm,
		top=1.5cm,
		bottom=2.5cm,
		includehead,
		includefoot
	}
	\addchap{Literaturverzeichnis}

	\setlength\columnsep{2em}
	\begin{multicols}{2}
		\begin{refcontext}[sorting=nyt]
			\renewcommand*{\bibfont}{\small\RaggedRight}
			\linespread{1.0}\selectfont
			\printbibliography[heading=none]
		\end{refcontext}
	\end{multicols}

	% DECLARATION
	\begin{declaration}
		\addchaptertocentry{\authorshipname}
		Ich erkläre hiermit gemä\ss\ \S~17 Abs.\,2 APO, dass ich die vorstehende {\thesistype}arbeit selbständig\\ verfasst und keine anderen als die angegebenen Quellen und Hilfsmittel benutzt habe.\\
		\bigskip
		\bigskip
		\\
		\begin{tabular}{@{}l@{}}
  		Bamberg, den \rule[-0.8em]{10em}{0.5pt}\\[2ex]
  		~
		\end{tabular}
		\hspace{\fill}%
		\begin{tabular}{@{}c@{}}
  			\rule[-0.8em]{20em}{0.5pt}\\[2ex]
  			\authorname
			\end{tabular}\hspace{\fill}
	\end{declaration}
\end{document}

	% TABLE OF CONTENTS
	\cleardoublepage

	\newgeometry{
		head=13.6pt,
		top=27.4mm,
		bottom=27.4mm,
		inner=24.8mm,
		outer=24.8mm,
		marginparsep=0mm,
		marginparwidth=0mm,
	}
	{
		\hypersetup{linkcolor=black}
		\tableofcontents % Prints the ToC entries
	}
	\restoregeometry

	% THESIS CONTENTS - CHAPTERS
	\mainmatter
	\pagestyle{thesis}

	\newcommand{\keyword}[1]{\textbf{#1}}
	\newcommand{\tabhead}[1]{\textbf{#1}}
	\newcommand{\code}[1]{\texttt{#1}}
	\newcommand{\file}[1]{\texttt{#1}}
	\newcommand{\option}[1]{\texttt{\itshape#1}}

	\graphicspath{{./figures/}{./examples/}}

	% CHAPTERS
	% !TeX program = lualatex
\chapter{Die Relevanz der Privatsphäre im 21. Jahrhundert - eine Einleitung} % Main chapter title
\label{Introduction} % For referencing the chapter elsewhere, use \ref{Introduction}

\enquote{We believe privacy is a fundamental human right} \cite{WWDC:20} - mit dieser Aussage betritt Craig Federighi, der Vizepräsident der Softwareentwicklung von Apple, einem Multimilliarden-Unternehmen der Software- und Hardwareentwicklung aus den Vereinigten Staaten, zur Vorstellung der 
neuen Privatsphäre-Richtliniendes Unternehmens zur sogenannten, firmeneigenen \enquote{Worldwide Developer Conference} die Bühne. \\ Und dieser Glaube ist nicht unbegründet: Mit der zunehmenden Digitalisierung in den letzten Jahrzehnten werden Menschen, und Internetnutzer im Spezifischen, 
häufiger vor dieses Dilemma gestellt. Entwicklungen wie dem Internet der Dinge (IoT), Industrie 4.0, \enquote{Smart Home}, Wearables (diverse Sensoren an der Kleidung selbst, Smartwatches, Fitnesstracker) uvm. ermöglichen neue (und zum Teil unerforschte) Angriffsmöglichkeiten für Cyberkriminalität \cite{BLB:18}.
Dies zeichnet sich vor allem durch den Anstieg der Höhe der Fallzahlen und der gleichzeitigen Abnahme der Aufklärungsquote aus: Im ersten Halbjahr 2018 wurden im Durchschnitt 13.000 Malware-Samples am Tag neu entdeckt \cite{GDB:18}, während diese Zahl 2017 noch bei [...] lag. \\ Ähnlich sieht es hier bei
dem Anstieg neuer Schadprogramme aus: Der Blog von G Data, einem deutschen Softwareunternehmen, welcher mehrfach für seine Sicherheitslösungen ausgezeichnet wurde \cite{GD+1}, beschreibt dabei, dass die Zahl der neuen Schadprogrammtypen seit 2007 einen Anstieg um das knapp 63-fache vermerken konnte \cite{GDB:17}.
	% !TeX program = lualatex
\chapter{Theoretischer Hintergrund} % Main chapter title
\label{Background} % For referencing the chapter elsewhere, use \ref{Background}

\section{Definition und Identifikation der Stakeholder in einem Softwareentwicklungsunternehmen}
Ziel dieser qualitativen Analyse ist es, die Denk- bzw. Sichtweisen von Stakeholdern in Softwareentwicklungsunternehmen zu erfassen und anhand dieser Ergebnisse, Schlüsse in Bezug auf die
eigene Privatsphäre dieser zu ziehen. Um ein grundlegendes Verständnis in den folgenden Kapiteln gewährleisten zu können ist es zunächst notwendig, diese Stakeholder zu definieren und anhand davon, 
potenzielle Stakeholder im Rahmen dieser Forschung zu identifizieren. \\
In den folgenden Unterkapiteln werden die untergliederten Interessensgruppen und ihre entsprechende Notwendigkeit näher erläutert. Die Begründung für die spezifische Wahl dieser besagten Gruppen wird im
nächsten Kapitel klassifiziert.


\subsection{Berufseinsteiger, Praktikanten und Werkstudenten}

\subsection{Softwareentwickler}

\subsection{Team-Leads, Projektleiter (und Geschäftsführer?)}

\subsection{Externe Ansprechpartner}

\section{DevOps und deren Einsatz in Softwareentwicklungsunternehmen}

\subsection{Interne Kommunikationstools}

\subsection{Projektmanagement- und Wiki-Software}

\subsection{Git}

\section{Aktueller Stand und Technologien?}
	% !TeX program = lualatex
\chapter{Forschungsmethode} % Main chapter title
\label{Research} % For referencing the chapter elsewhere, use \ref{Research}

Dabei wurde die Vorgehensweise zur Durchführung von Experteninterviews gewählt: Hierzu wurden einzelne Softwareentwickler von diversen Unternehmen sorgfältig
anhand ihrer Qualifikation, Berufserfahrung und der Position im Unternehmen gewählt - um ein möglichst weites Spektrum an unterschiedlichen Personengruppen abzudecken, wurden diese in Berufseinsteiger,
welche kaum bis relativ wenig Berufserfahrung vorweisen können, mehrjährig Festangestellte Softwareentwickler und Team-Leads bzw. Projektleiter, welche zusätzlich noch verantwortlich für Entwicklergruppen sind,
untergliedert.

\section{Definition und Identifikation der Stakeholder in einem Softwareentwicklungsunternehmen}
Ziel dieser qualitativen Analyse ist es, die Denk- bzw. Sichtweisen von Stakeholdern in Softwareentwicklungsunternehmen zu erfassen und anhand dieser Ergebnisse, Schlüsse in Bezug auf die
eigene Privatsphäre dieser zu ziehen. Um ein grundlegendes Verständnis in den folgenden Kapiteln gewährleisten zu können ist es zunächst notwendig, diese Stakeholder zu definieren und anhand davon, 
potenzielle Stakeholder im Rahmen dieser Forschung zu identifizieren. \\ Als Stakeholder werden jene \enquote{Personen, Gruppen oder Institutionen bezeichnet, die von den Aktivitäten eines Unternehmens 
direkt oder indirekt betroffen sind oder [...] ein Interesse an diesen [...] haben} \cite{Fle:16} - dies kann von Kunden und Lieferanten bis zu den eigenen Mitarbeitern und Eigentümern reichen \cite{Fle:16}. 
Im Falle dieser Analyse wird die Sichtweise der einzelnen Stakeholder auf die eigene Privatsphäre im angestellten Unternehmen betrachtet. \\ Gemessen wird dieser Aspekt durch die Relevanz, wie viel Wert die befragten
Softwareentwickler beispielsweise darauf legen, personenbezogene Daten in einem möglichst weiten Spektrum für einen möglichst langen Zeitraum einsehen zu können und auf der Kehrseite, wie weit sie bereit sind, eigene
personenbezogene Daten preiszugeben bzw. mit Kollegen und anderen Stakeholdern zu teilen, um ein gut funktionierendes Glied eines Teams oder eines Unternehmens sein zu können.

In den folgenden Unterkapiteln werden die untergliederten Interessensgruppen und ihre entsprechende Notwendigkeit näher erläutert. Die Begründung für die spezifische Wahl dieser besagten Gruppen wird im
nächsten Kapitel klassifiziert.

\subsection{Berufseinsteiger, Praktikanten und Werkstudenten}
Die primäre Gruppe zeichnet sich durch Softwareentwickler aus, welche kaum bis wenig Berufserfahrung besitzen und eventuell geringere Qualifikationen, als die anderen Interessensgruppen aufweisen könnten.
Diese Merkmale finden sich in Berufseinsteigern wieder, welche vor Kurzem aus einer abgeschlossenen Ausbildung oder eines abgeschlossenen Studiums stammen und nun im Beruf Softwareentwickler sind. Selbe Punkte
lassen sich allerdings auch auf (längerfristige) Praktikanten und Werkstudenten, welche neben ihrem Studium in einem Unternehmen bis zu 20 Stunden in der Woche tätig sind, übertragen - es ist also wenig Berufserfahrung
vorhanden, aber grundlegende Kenntnisse in der Informatik, vor allem in Bezug auf DevOps, welche im nächsten Unterkapitel näher erläutert werden, lassen sich in diesen Interessensgruppen wiederfinden.

\subsection{Softwareentwickler}

\subsection{Team-Leads, Projektleiter (und Geschäftsführer?)}

\subsection{Externe Ansprechpartner}

\section{DevOps und deren Einsatz in Softwareentwicklungsunternehmen}

\subsection{Interne Kommunikationstools}

\subsection{Projektmanagement- und Wiki-Software}

\subsection{Git}

\section{Aktueller Stand und Technologien?}
	% !TeX program = lualatex
\chapter{Ergebnisse} % Main chapter title
\label{Results} % For referencing the chapter elsewhere, use \ref{Results}

Anhand der Expertenbefragungen konnten insgesamt 18 Kategorien zu den verschiedenen Sichtweisen herausausgearbeitet werden. Hierbei wird zwischen
der Präferenz der Privatsphäre im Allgemeinen oder der Benutzerfreundlichkeit bzw. Einfachheit unterschieden: Kategorien, in welchen Experten überwiegend
(>50\% der Experten sind der selben Ansicht) angegeben haben, die Privatsphäre sei ihnen wichtiger, werdem dem Kapitel \ref{privacy} zugeordnet, wohingegen die Kategorien, in welchen die Privatsphäre
aufgrund der Benutzerfreundlichkeit und Einfachheit vernachlässigt werden konnte, dem Kapitel \ref{noprivacy} zugerechnet werden. Im Anschluss werden die Ergebnisse,
in welchen kein einstimmiges Ergebnis (bei sieben Befragten ergibt dies: Weniger als 28\% der Befragten haben die selbe Meinung) erzielt werden konnte, in Kapitel \ref{noclearresult} thematisiert.

\section{Kategorien mit einer Präferenz für die Privatsphäre} \label{privacy}
\textbf{Interesse an persönlichen Daten in Log-Dateien:} \newline
Dieser Punkt befasst sich mit dem allgemeinen Interesse an persönlichen Daten, welche in Form von Benutzernamen, Zeitstempeln, IDs etc. in Log-Dateien sämtlicher Art auftreten können. Hier haben die Befragten einstimmig
entschieden, kein Interesse an diesen zu haben und ein mögliches Entfernen in der Zukunft willkommen zu heißen. Es wurde lediglich angegeben, dass in bestimmten Fällen (Zusammenarbeit an einem Projekt, 
in Testfällen von Code, Servern oder Applikationen oder auf Anfrage) eine Nachverfolgung auf Wunsch eines Kunden oder eines Auftrags erfolgen muss. Für die Befragten ist es nur wichtig, den Ablauf und auftretende Fehler 
von Servern, Applikationen oder Code im Allgemeinen nachverfolgen zu können - ein Interesse an bestimmten Personen oder Zeitstempeln ist nicht vorhanden, weswegen in dieser Kategorie die Privatsphäre über der Nutzerfreundlichkeit
steht. \newline \newline
\textbf{Bevorzugte Aufbewahrungsdauer von Log-Dateien:} \newline
Als Weiterleitung der Frage über die tatsächliche Aufbewahrungsdauer von Log-Dateien in Kapitel \ref{noprivacy} wurden in dieser Kategorie die Experten dazu angehalten, ihre persönliche Präferenz zur Dauer anzugeben. Hier haben alle 
Befragten angegeben, eine temporäre Aufbewahrungsdauer zu bevorzugen. Diese könne anhand der Lebensdauer von Projekten, der Lebensdauer von Programmcode, den letzten x Builds, einer festgelegten Zeit (z.B. zwei Wochen)
oder der Existenz von gemeldeten Fehlern deklariert werden. In der Regel sind sich die Befragten einig, eine Dauer von maximal wenigen Monaten zu bevorzugen. Diese Sichtweisen stellen einen Gegensatz zur tatsächlichen Aufbewahrungsdauer von
Log-Dateien in Softwareentwicklungsunternehmen dar, welche in Kapitel \ref{noprivacy} angesprochen werden.

\section{Kategorien mit einer Präferenz für die Benutzerfreundlichkeit und Einfachheit} \label{noprivacy}
\textbf{Verarbeitete Daten zur Unterstützung der Softwareentwicklung in DevOps-Tools:} \newline
Zu Beginn der Interviews haben alle Befragten angegeben, mit persönlichen Daten in Berührung zu kommen und diese auch zu verarbeiten. Dabei haben alle vier Softwareentwickler angegeben, die Nutzeraktivitäten und den Programmcode von Kollegen 
einsehen und verarbeiten zu können. Dies erfolge zur Nachvollziehbarkeit der Builds und Deployments, des Programmcodes selbst, z.B. durch Git Commits und zur allgemeinen Fehlerbehebung bei fehlgeschlagenen Builds bzw. fehlerhaftem Programmcode. 
Zwei Befragte haben zudem angegeben, Zugriff auf sensible Kundendaten in Form von Klarnamen, Anschriften, E-Mail Adressen uvm. zu besitzen: Dies ist bei jenen der Fall, die häufig mit externen Ansprechpartnern und Unternehmen in Kontakt treten 
und mit diesen zusammenarbeiten müssen. \newline \newline
\textbf{Tatsächliche Aufbewahrungsdauer von Log-Dateien:} \newline
Da ein sehr großer Anteil der Befragten von über 85\% (vgl. Tabelle \ref{tab:generaldata}) angegeben haben, Log-Dateien aktiv zu nutzen, ist es für den weiteren Verlauf der Analyse erforderlich, die tatsächliche Aufbewahrungsdauer der genutzten Log-Dateien 
im Unternehmen zu spezifizieren. Hier haben alle Befragten angegeben, dass alle Log-Dateien in der Regel für eine unbegrenzte bzw. unbestimmte Dauer aufbewahrt werden. Die Stakeholder haben zudem meist keine Kenntnis über die genaue Aufbewahrungsdauer und 
beziehen sich bei ihrer Aussage auf die Existenz von sehr alten Log-Dateien (über 5 Jahre). Die Ausnahme dieser Regel betrifft manuell durchgeführte Bereinigungen von Speicher und eingeleitete Löschung spezifischer Log-Dateien: Bei einem Befragten kam es vor,
dass das System einen Absturz erlitten hat, welche die Log-Dateien unbenutzbar gemacht haben, wohingegen bei einem anderen der Serverspeicher geleert werden musste und dieses Vorgehen sämtliche Log-Dateien gelöscht hat. \newline Aus diesen Aussagen lässt sich 
ableiten, dass ein Vorgehen zur automatisierten Löschung von Log-Dateien bei den Befragten bisher keine Anwendung gefunden hat. \newline \newline
\textbf{Interesse an anderen persönlichen Daten in DevOps-Tools:} \\
\textbf{Gründe für das Interesse an Zeitstempeln in DevOps-Tools:} \\
\textbf{Interesse an persönlichen Daten in Review-Tools:} \\
\textbf{Interesse an persönlichen Daten in Administrationstools:} \\
\textbf{Detailgrad von Issues in Ticketverwaltungstools:} \\
\textbf{Personenbezug von Issue-Kommentaren in Ticketverwaltungstools:} \\
\textbf{Allgemeine Auswertungen von persönlichen Daten aus DevOps-Tools:} \\
\textbf{Relevanz von persönlichen Daten in Builds:} \\
\textbf{Interesse an Log-Dateien:} \\
\textbf{Mögliche Verarbeitungsinteressen von persönlichen Daten in anderen Unternehmen:} \\

\section{Kategorien mit unstimmigen Ergebnissen} \label{noclearresult}
\textbf{Detailgrad von Zeitstempeln in DevOps-Tools:} \\
\textbf{Sonstige Verarbeitungsinteressen von persönlichen Daten:} \\

\section{Vorgeschlagene Alternativen}

\section{Zusammenstellung der Ergebnisse}





	% !TeX program = lualatex
\chapter{Diskussion der Ergebnisse} % Main chapter title
\label{Discussion} % For referencing the chapter elsewhere, use \ref{Discussion}

\section{Implikationen in der Theorie}

\section{Implikationen in der Praxis}

\section{Limitation der Arbeit}
	% !TeX program = lualatex
\chapter{Fazit und Ausblick} % Main chapter title
\label{Conclusion} % For referencing the chapter elsewhere, use \ref{Conclusion}

Im Umfang dieser Analyse wurde die Privatsphäre in Verbindung mit Stakeholdern und DevOps in Softwareentwicklungsunternehmen präsentiert
und anhand Belege aktueller Forschungen ein erster Versuch einer praktischen Feldanalyse gewagt. Anhand der Expertenbefragungen konnten so einzelne
Kategorisierungen herausgearbeitet werden, welche entweder einer Präferenz zur Einfachheit und Benutzerfreundlichkeit oder der Privatsphäre im Arbeitsalltag
und -ablauf zugeordnet wurden. Die einzelnen Kategorien deckten sich dabei mit den Annahmen aus der Forschung. \newline
Da eine so geringe Anzahl an gewählten Experten für eine Analyse jedoch nicht als vollständig repräsentativ angesehen werden kann, ist diese Arbeit als eine
Unterstützung der aktuellen Forschung in Bezug auf die Privatsphäre der Stakeholder und deren Sichtweisen am Arbeitsplatz in Softwareentwicklungsunternehmen
zu betrachten. \newline \newline
Nun bleibt zu hoffen, dass zusätzlich zu den aktuellen Entwicklungen, welche mit der zunehmenden Digitalisierung Eintritt in den Alltag aller finden, neue Entdeckungen
auch in Hinsicht auf Privatsphäre in der Zukunft kritisch begutachtet werden. Hilfreich ist es allemal, wenn \enquote{Big Player} der Software- bzw. Hardwareentwicklung
ein Vorbild repräsentieren und die Relevanz der Privatsphäre immer wieder betonen. \newline Vielleicht überlegt es sich manch einer auf diese Weise zweimal, eine Kundenkarte
wie Payback, der DeutschlandCard o.Ä. zu beantragen und beim Einkaufen zu verwenden.

	% APPENDICES
	\addtocontents{toc}{\string\def\string\chaptername{Appendix}}
	\appendix
	\renewcommand{\thesection}{\thechapter.\arabic{section}}
	\renewcommand{\thesubsection}{\thesection.\arabic{subsection}}
	\renewcommand{\thesubsubsection}{\thesubsection.\arabic{subsubsection}}

	% Appendix A
\chapter{Verbundene Arbeiten}

\section{Stellenausschreibung zur Befragung von Softwareentwicklern bzw. -administratoren} \label{ausschreibung}
Im Rahmen einer Abschlussarbeit, welche sich mit der qualitativen Analyse von Stakeholdern in einem IT-Unternehmen zu Sichtweisen auf Privatsphäre befasst, sind wir auf der Suche nach potenziellen Befragungen von Softwareentwicklern/-administratoren. Sie sollten bereits
\begin{itemize}
    \item in einem Softwareentwicklungsunternehmen gearbeitet haben
    \item Erfahrungen mit der Arbeit im Team haben
    \item mit einem Versionsverwaltungs-Tool (z.B. Git, SVN etc.) gearbeitet haben
    \item grobe Kenntnisse über interne Abläufe besitzen
\end{itemize}
Melden Sie sich bei Interesse bitte bei mir, \textbf{Samet Akcabay}, unter der E-Mail-Adresse \textbf{samet-murat.akcabay@stud.uni-bamberg.de}!

\section{Datenschutzerklärung} \label{erklaerung}
\textbf{Datenschutzerklärung zur Befragung im Rahmen der Abschlussarbeit \enquote{Qualitative Analyse von Stakeholdern in einem Softwareentwicklungsunternehmen zu Sichtweisen auf Privatsphäre bei Softwareentwicklern bzw. -administratoren}} \newline \newline
Wir, der Lehrstuhl für Privatsphäre und Sicherheit in Informationssystemen, legen großen Wert auf Datenschutz und das Recht auf informationelle Selbstbestimmung. Aufgrund dessen möchten wir Sie mit dieser Datenschutzerklärung über die Erhebung, Speicherung und Verarbeitung der Daten, die im Zusammenhang mit der Abschlussarbeit „Qualitative Analyse von Stakeholdern in einem Softwareentwicklungsunternehmen zu Sichtweisen auf Privatsphäre bei Softwareentwicklern bzw. -administratoren“ erhoben werden, informieren. \newline \newline
Alle personenbezogenen Daten werden in keinster Weise betrachtet, verarbeitet oder an eine andere Instanz weitergeleitet. Bei der Befragung geht es lediglich darum, die Interessen in verschiedenen Unternehmen sowie von verschiedenen Interessensgruppen zu analysieren. Jede einzelne Angabe in jeder Befragung wird im Anschluss zu allgemeinen Punkten kategorisiert und zu Mustern zusammengefasst. Nach erfolgreicher Verarbeitung der nicht-personenbezogenen Daten werden sämtliche Aufnahmen und Transkriptionen unwiderruflich gelöscht. \newline \newline
Die befragten Personen willigen bei Teilnahme an der Befragung im Rahmen der Abschlussarbeit dieser Datenschutzerklärung zu. Sie haben jederzeit das Recht, Ihre Angaben zu widerrufen. In diesem Fall werden alle von Ihnen gespeicherten Daten unwiderruflich gelöscht und eine Verarbeitung dieser findet nicht statt. \newline \newline
Sie können jederzeit den Lehrstuhl für Privatsphäre und Sicherheit in Informationssystemen oder den Verfasser der Abschlussarbeit bei Fragen oder Anregungen kontaktieren: \newline \newline
\textbf{Universität Bamberg\newline
Lehrstuhl für Privatsphäre und Sicherheit in Informationssystemen An der Weberei 5\newline
96047 Bamberg\newline \newline
Raum:} WE5/05.064\newline
\textbf{Sekretariat:} WE5/05.063\newline 
\textbf{Telefon:} +49 (0)951–863 2661\newline \newline
\textbf{Samet Akcabay\newline
An der Spinnerei 11 Zimmer-Nr. D29 \newline96047 Bamberg\newline \newline
Telefon:} +49 (0)172-315 5296 \newline
\textbf{E-Mail:} samet-murat.akcabay@stud.uni-bamberg.de


\section{Online-Formular zum Ausfüllen über Microsoft Forms} \label{forms}
\begin{enumerate}
    \item Welche Rolle übernehmen Sie im Unternehmen?
    \item Administrieren Sie auch Server?
    \item Welche Daten verarbeiten Sie allgemein bei der Softwareentwicklung? Welche Daten schauen Sie sich explizit mithilfe von DevOps-Tools an, die Sie bei der Softwareentwicklung unterstützen?
    \item Wo liegen diese Daten, die von Ihnen verarbeitet werden? Finden sich unter diesen auch personenbezogene Daten?
    \item Was bedeutet für Sie allgemein "Daten verarbeiten"? Welche Abläufe bringen Sie damit in Verbindung?
    \item Wie genau können Sie diese von Ihnen genannten Daten einsehen?
    \item Schauen Sie sich bestimmte Daten explizit an bzw. legen Sie Wert darauf, dass bestimmte Daten gegeben sind?
    \item Schauen Sie sich auch explizit an, wer an welchen Issues/Builds (falls vorhanden) in welchem Projekt gearbeitet hat?
    \item Reviewen Sie auch Code von anderen Entwicklern? Falls ja, was genau schauen Sie sich beim verwendeten Review-Tool an (z.B. Name des Reviewers, Zeitstempel, ID etc.)? Worauf legen Sie besonders wert?
    \item Allgemein gefragt: Was wäre für Sie eine angemessene Zeit zur Aufbewahrung von personenbezogenen Daten? Bitte begründen Sie Ihre Antwort.
    \item Wie lange werden die von Ihnen genannten Daten bei Ihnen im Unternehmen tatsächlich aufbewahrt?
    \item Zurück zu Issues/Builds: Schauen Sie sich auch an, wann/ob/von wem der Issue erstellt/bearbeitet/kommentiert wurde?
    \item Welche Interessen verfolgen Sie generell, wenn Sie sich Zeitstempel in DevOps-Tools ansehen? Worauf legen Sie besonders wert?
    \item Von der anderen Seite aus betrachtet: Worauf könnten Sie verzichten?
    \item Was glauben Sie, zu welchem Zweck man diese Tools verwendet? (z.B. Nachvollziehbarkeit, Leistungsbeurteilung, Qualitätssicherung etc.)
    \item Könnten Sie sich andere Auswertungen aus den Daten vorstellen? (Auch, was Sie persönlich noch nicht selbst erlebt/gemacht haben, sondern welche im Interesse des Unternehmens liegen)
    \item Verwenden/arbeiten Sie mit Log-Dateien?
    \item Falls ja: Finden Sie es wichtig, dass Benutzername und Zeitstempel in den Log-Dateien vorhanden sind? Falls nicht, worauf würden Sie stattdessen setzen?
    \item Wie lange werden Log-Dateien bei Ihnen aufbewahrt? (z.B. von Tickets, Builds, Issues etc.)
    \item Fällt Ihnen noch etwas bisher nicht genanntes ein, worauf Sie wert legen und was Sie sich ansehen?
    \item Fallen Ihnen andere Verarbeitungsinteressen bzw. Auswertungsansätze ein, die andere Firmen haben könnten?

\end{enumerate}
	% Appendix B

\chapter{Interviewleitfaden} \label{interview}

% \begin{table*}[t]
%     \caption{Verwendeter Leitfaden für die Befragung zu den Sichtweisen auf die Privatsphäre in Softwareentwicklungsunternehmen (angelehnt an \cite{Helfferich:2011aa})}
%     \label{tab:leitfaden}
%     \centering
%     \small % use smaller fontsize in the table
%     {\renewcommand{\arraystretch}{2} % increase vertical space between rows
%     \begin{tabularx}{\linewidth}{@{}llllX@{}} % @{} omits outer horizontal margins, the "X" column uses up all remaining available space 
%       \toprule
%       Leitfrage & Check & Konkrete Fragen & Aufrechterhaltungs-/Steuerungsfragen \\
%       \midrule
%         \enquote{Welche Daten verarbeiten Sie?}                                                                             & Desinteresse                  & Persönliche Daten in DevOps-Tools             \\
%         \enquote{Was schauen Sie sich konkret an?}                                                                          & Nutzeraktivität               & Persönliche Daten in Administrationstools     \\
%         \enquote{In welcher Auflösung brauchen Sie den Zeitstempel?}                                                        & Desinteresse                  & Nutzung von Tools zur Administration          \\
%         \enquote{Gibt es etwas anderes, was Sie sich ansehen?}                                                              & Desinteresse                  & Persönliche Daten in DevOps-Tools             \\
%         \enquote{Schauen Sie sich auch an, wer spezifisch was gebaut hat?}                                                  & Benutzername                  & Persönliche Daten in DevOps-Tools             \\
%         \enquote{Was schauen Sie sich beim Review-Tool an?}                                                                 & Benutzername                  & Persönliche Daten in DevOps-Tools             \\
%         \enquote{Wie lange ist es für Sie wichtig, dass diese Daten aufbewahrt werden?}                                     & Benutzername                  & Persönliche Daten in DevOps-Tools             \\
%         \enquote{Wie lange werden diese tatsächlich aufbewahrt?}                                                            & Benutzername                  & Persönliche Daten in DevOps-Tools             \\
%         \enquote{Was für Daten schauen Sie sich in Issues an?}                                                              & Benutzername                  & Persönliche Daten in DevOps-Tools             \\
%         \enquote{Schauen Sie sich auch an, von wem ein Kommentar editiert wurde?}                                           & Benutzername                  & Persönliche Daten in DevOps-Tools             \\
%         \enquote{Welche Interessen verfolgen Sie generell bei Zeitstempeln in DevOps-Tools?}                                & Benutzername                  & Persönliche Daten in DevOps-Tools             \\
%         \enquote{Welche Auswertungen, die sie selbst nicht machen, können sie sich sonst mit diesen Daten vorstellen?}      & Benutzername                  & Persönliche Daten in DevOps-Tools             \\
%         \enquote{Schauen Sie sich auch Logs an?}                                                                            & Benutzername                  & Persönliche Daten in DevOps-Tools             \\
%         \enquote{Welche Auswertungen, die sie selbst nicht machen, können sie sich sonst mit diesen Daten vorstellen?}      & Benutzername                  & Persönliche Daten in DevOps-Tools             \\
%         \enquote{Ist es für Sie auch wichtig, wer wann was gemacht hat?}                                                    & Benutzername                  & Persönliche Daten in DevOps-Tools             \\
%         \enquote{Wissen Sie, wie lange diese Log-Dateien aufbewahrt werden?}                                                & Benutzername                  & Persönliche Daten in DevOps-Tools             \\
%         \enquote{Welche Tools verwenden Sie zur Administration?}                                                            & Benutzername                  & Persönliche Daten in DevOps-Tools             \\
%         \enquote{Haben Sie da die Möglichkeit einzusehen, wer wann was gemacht hat?}                                        & Benutzername                  & Persönliche Daten in DevOps-Tools             \\
%         \enquote{Warum ist das \enquote{wer} für Sie relevant?}                                                             & Benutzername                  & Persönliche Daten in DevOps-Tools             \\
%         \enquote{Können Sie sich andere Verarbeitungsinteressen vorstellen?}                                                & Benutzername                  & Persönliche Daten in DevOps-Tools             \\
%       \bottomrule
%     \end{tabularx}
%     }
%   \end{table*}
	%\include{appendices/appendixC}

	% BIBLIOGRAPHY
	\newgeometry{
		inner=2cm,
		outer=2cm, 
		marginparwidth=0cm,
		marginparsep=0mm,
		bindingoffset=.5cm,
		top=1.5cm,
		bottom=2.5cm,
		includehead,
		includefoot
	}
	\addchap{Literaturverzeichnis}

	\setlength\columnsep{2em}
	\begin{multicols}{2}
		\begin{refcontext}[sorting=nyt]
			\renewcommand*{\bibfont}{\small\RaggedRight}
			\linespread{1.0}\selectfont
			\printbibliography[heading=none]
		\end{refcontext}
	\end{multicols}

	% DECLARATION
	\begin{declaration}
		\addchaptertocentry{\authorshipname}
		Ich erkläre hiermit gemä\ss\ \S~17 Abs.\,2 APO, dass ich die vorstehende {\thesistype}arbeit selbständig\\ verfasst und keine anderen als die angegebenen Quellen und Hilfsmittel benutzt habe.\\
		\bigskip
		\bigskip
		\\
		\begin{tabular}{@{}l@{}}
  		Bamberg, den \rule[-0.8em]{10em}{0.5pt}\\[2ex]
  		~
		\end{tabular}
		\hspace{\fill}%
		\begin{tabular}{@{}c@{}}
  			\rule[-0.8em]{20em}{0.5pt}\\[2ex]
  			\authorname
			\end{tabular}\hspace{\fill}
	\end{declaration}
\end{document}

	% TABLE OF CONTENTS
	\cleardoublepage

	\newgeometry{
		head=13.6pt,
		top=27.4mm,
		bottom=27.4mm,
		inner=24.8mm,
		outer=24.8mm,
		marginparsep=0mm,
		marginparwidth=0mm,
	}
	{
		\hypersetup{linkcolor=black}
		\tableofcontents % Prints the ToC entries
	}
	\restoregeometry

	% THESIS CONTENTS - CHAPTERS
	\mainmatter
	\pagestyle{thesis}

	\newcommand{\keyword}[1]{\textbf{#1}}
	\newcommand{\tabhead}[1]{\textbf{#1}}
	\newcommand{\code}[1]{\texttt{#1}}
	\newcommand{\file}[1]{\texttt{#1}}
	\newcommand{\option}[1]{\texttt{\itshape#1}}

	\graphicspath{{./figures/}{./examples/}}

	% CHAPTERS
	% !TeX program = lualatex
\chapter{Die Relevanz der Privatsphäre im 21. Jahrhundert - eine Einleitung} % Main chapter title
\label{Introduction} % For referencing the chapter elsewhere, use \ref{Introduction}

\enquote{We believe privacy is a fundamental human right} \cite{WWDC:20} - mit dieser Aussage betritt Craig Federighi, der Vizepräsident der Softwareentwicklung von Apple, einem Multimilliarden-Unternehmen der Software- und Hardwareentwicklung aus den Vereinigten Staaten, zur Vorstellung der 
neuen Privatsphäre-Richtliniendes Unternehmens zur sogenannten, firmeneigenen \enquote{Worldwide Developer Conference} die Bühne. \\ Und dieser Glaube ist nicht unbegründet: Mit der zunehmenden Digitalisierung in den letzten Jahrzehnten werden Menschen, und Internetnutzer im Spezifischen, 
häufiger vor dieses Dilemma gestellt. Entwicklungen wie dem Internet der Dinge (IoT), Industrie 4.0, \enquote{Smart Home}, Wearables (diverse Sensoren an der Kleidung selbst, Smartwatches, Fitnesstracker) uvm. ermöglichen neue (und zum Teil unerforschte) Angriffsmöglichkeiten für Cyberkriminalität \cite{BLB:18}.
Dies zeichnet sich vor allem durch den Anstieg der Höhe der Fallzahlen und der gleichzeitigen Abnahme der Aufklärungsquote aus: Im ersten Halbjahr 2018 wurden im Durchschnitt 13.000 Malware-Samples am Tag neu entdeckt \cite{GDB:18}, während diese Zahl 2017 noch bei [...] lag. \\ Ähnlich sieht es hier bei
dem Anstieg neuer Schadprogramme aus: Der Blog von G Data, einem deutschen Softwareunternehmen, welcher mehrfach für seine Sicherheitslösungen ausgezeichnet wurde \cite{GD+1}, beschreibt dabei, dass die Zahl der neuen Schadprogrammtypen seit 2007 einen Anstieg um das knapp 63-fache vermerken konnte \cite{GDB:17}.
	% !TeX program = lualatex
\chapter{Theoretischer Hintergrund} % Main chapter title
\label{Background} % For referencing the chapter elsewhere, use \ref{Background}

\section{Definition und Identifikation der Stakeholder in einem Softwareentwicklungsunternehmen}
Ziel dieser qualitativen Analyse ist es, die Denk- bzw. Sichtweisen von Stakeholdern in Softwareentwicklungsunternehmen zu erfassen und anhand dieser Ergebnisse, Schlüsse in Bezug auf die
eigene Privatsphäre dieser zu ziehen. Um ein grundlegendes Verständnis in den folgenden Kapiteln gewährleisten zu können ist es zunächst notwendig, diese Stakeholder zu definieren und anhand davon, 
potenzielle Stakeholder im Rahmen dieser Forschung zu identifizieren. \\
In den folgenden Unterkapiteln werden die untergliederten Interessensgruppen und ihre entsprechende Notwendigkeit näher erläutert. Die Begründung für die spezifische Wahl dieser besagten Gruppen wird im
nächsten Kapitel klassifiziert.


\subsection{Berufseinsteiger, Praktikanten und Werkstudenten}

\subsection{Softwareentwickler}

\subsection{Team-Leads, Projektleiter (und Geschäftsführer?)}

\subsection{Externe Ansprechpartner}

\section{DevOps und deren Einsatz in Softwareentwicklungsunternehmen}

\subsection{Interne Kommunikationstools}

\subsection{Projektmanagement- und Wiki-Software}

\subsection{Git}

\section{Aktueller Stand und Technologien?}
	% !TeX program = lualatex
\chapter{Forschungsmethode} % Main chapter title
\label{Research} % For referencing the chapter elsewhere, use \ref{Research}

Dabei wurde die Vorgehensweise zur Durchführung von Experteninterviews gewählt: Hierzu wurden einzelne Softwareentwickler von diversen Unternehmen sorgfältig
anhand ihrer Qualifikation, Berufserfahrung und der Position im Unternehmen gewählt - um ein möglichst weites Spektrum an unterschiedlichen Personengruppen abzudecken, wurden diese in Berufseinsteiger,
welche kaum bis relativ wenig Berufserfahrung vorweisen können, mehrjährig Festangestellte Softwareentwickler und Team-Leads bzw. Projektleiter, welche zusätzlich noch verantwortlich für Entwicklergruppen sind,
untergliedert.

\section{Definition und Identifikation der Stakeholder in einem Softwareentwicklungsunternehmen}
Ziel dieser qualitativen Analyse ist es, die Denk- bzw. Sichtweisen von Stakeholdern in Softwareentwicklungsunternehmen zu erfassen und anhand dieser Ergebnisse, Schlüsse in Bezug auf die
eigene Privatsphäre dieser zu ziehen. Um ein grundlegendes Verständnis in den folgenden Kapiteln gewährleisten zu können ist es zunächst notwendig, diese Stakeholder zu definieren und anhand davon, 
potenzielle Stakeholder im Rahmen dieser Forschung zu identifizieren. \\ Als Stakeholder werden jene \enquote{Personen, Gruppen oder Institutionen bezeichnet, die von den Aktivitäten eines Unternehmens 
direkt oder indirekt betroffen sind oder [...] ein Interesse an diesen [...] haben} \cite{Fle:16} - dies kann von Kunden und Lieferanten bis zu den eigenen Mitarbeitern und Eigentümern reichen \cite{Fle:16}. 
Im Falle dieser Analyse wird die Sichtweise der einzelnen Stakeholder auf die eigene Privatsphäre im angestellten Unternehmen betrachtet. \\ Gemessen wird dieser Aspekt durch die Relevanz, wie viel Wert die befragten
Softwareentwickler beispielsweise darauf legen, personenbezogene Daten in einem möglichst weiten Spektrum für einen möglichst langen Zeitraum einsehen zu können und auf der Kehrseite, wie weit sie bereit sind, eigene
personenbezogene Daten preiszugeben bzw. mit Kollegen und anderen Stakeholdern zu teilen, um ein gut funktionierendes Glied eines Teams oder eines Unternehmens sein zu können.

In den folgenden Unterkapiteln werden die untergliederten Interessensgruppen und ihre entsprechende Notwendigkeit näher erläutert. Die Begründung für die spezifische Wahl dieser besagten Gruppen wird im
nächsten Kapitel klassifiziert.

\subsection{Berufseinsteiger, Praktikanten und Werkstudenten}
Die primäre Gruppe zeichnet sich durch Softwareentwickler aus, welche kaum bis wenig Berufserfahrung besitzen und eventuell geringere Qualifikationen, als die anderen Interessensgruppen aufweisen könnten.
Diese Merkmale finden sich in Berufseinsteigern wieder, welche vor Kurzem aus einer abgeschlossenen Ausbildung oder eines abgeschlossenen Studiums stammen und nun im Beruf Softwareentwickler sind. Selbe Punkte
lassen sich allerdings auch auf (längerfristige) Praktikanten und Werkstudenten, welche neben ihrem Studium in einem Unternehmen bis zu 20 Stunden in der Woche tätig sind, übertragen - es ist also wenig Berufserfahrung
vorhanden, aber grundlegende Kenntnisse in der Informatik, vor allem in Bezug auf DevOps, welche im nächsten Unterkapitel näher erläutert werden, lassen sich in diesen Interessensgruppen wiederfinden.

\subsection{Softwareentwickler}

\subsection{Team-Leads, Projektleiter (und Geschäftsführer?)}

\subsection{Externe Ansprechpartner}

\section{DevOps und deren Einsatz in Softwareentwicklungsunternehmen}

\subsection{Interne Kommunikationstools}

\subsection{Projektmanagement- und Wiki-Software}

\subsection{Git}

\section{Aktueller Stand und Technologien?}
	% !TeX program = lualatex
\chapter{Ergebnisse} % Main chapter title
\label{Results} % For referencing the chapter elsewhere, use \ref{Results}

Anhand der Expertenbefragungen konnten insgesamt 18 Kategorien zu den verschiedenen Sichtweisen herausausgearbeitet werden. Hierbei wird zwischen
der Präferenz der Privatsphäre im Allgemeinen oder der Benutzerfreundlichkeit bzw. Einfachheit unterschieden: Kategorien, in welchen Experten überwiegend
(>50\% der Experten sind der selben Ansicht) angegeben haben, die Privatsphäre sei ihnen wichtiger, werdem dem Kapitel \ref{privacy} zugeordnet, wohingegen die Kategorien, in welchen die Privatsphäre
aufgrund der Benutzerfreundlichkeit und Einfachheit vernachlässigt werden konnte, dem Kapitel \ref{noprivacy} zugerechnet werden. Im Anschluss werden die Ergebnisse,
in welchen kein einstimmiges Ergebnis (bei sieben Befragten ergibt dies: Weniger als 28\% der Befragten haben die selbe Meinung) erzielt werden konnte, in Kapitel \ref{noclearresult} thematisiert.

\section{Kategorien mit einer Präferenz für die Privatsphäre} \label{privacy}
\textbf{Interesse an persönlichen Daten in Log-Dateien:} \newline
Dieser Punkt befasst sich mit dem allgemeinen Interesse an persönlichen Daten, welche in Form von Benutzernamen, Zeitstempeln, IDs etc. in Log-Dateien sämtlicher Art auftreten können. Hier haben die Befragten einstimmig
entschieden, kein Interesse an diesen zu haben und ein mögliches Entfernen in der Zukunft willkommen zu heißen. Es wurde lediglich angegeben, dass in bestimmten Fällen (Zusammenarbeit an einem Projekt, 
in Testfällen von Code, Servern oder Applikationen oder auf Anfrage) eine Nachverfolgung auf Wunsch eines Kunden oder eines Auftrags erfolgen muss. Für die Befragten ist es nur wichtig, den Ablauf und auftretende Fehler 
von Servern, Applikationen oder Code im Allgemeinen nachverfolgen zu können - ein Interesse an bestimmten Personen oder Zeitstempeln ist nicht vorhanden, weswegen in dieser Kategorie die Privatsphäre über der Nutzerfreundlichkeit
steht. \newline \newline
\textbf{Bevorzugte Aufbewahrungsdauer von Log-Dateien:} \newline
Als Weiterleitung der Frage über die tatsächliche Aufbewahrungsdauer von Log-Dateien in Kapitel \ref{noprivacy} wurden in dieser Kategorie die Experten dazu angehalten, ihre persönliche Präferenz zur Dauer anzugeben. Hier haben alle 
Befragten angegeben, eine temporäre Aufbewahrungsdauer zu bevorzugen. Diese könne anhand der Lebensdauer von Projekten, der Lebensdauer von Programmcode, den letzten x Builds, einer festgelegten Zeit (z.B. zwei Wochen)
oder der Existenz von gemeldeten Fehlern deklariert werden. In der Regel sind sich die Befragten einig, eine Dauer von maximal wenigen Monaten zu bevorzugen. Diese Sichtweisen stellen einen Gegensatz zur tatsächlichen Aufbewahrungsdauer von
Log-Dateien in Softwareentwicklungsunternehmen dar, welche in Kapitel \ref{noprivacy} angesprochen werden.

\section{Kategorien mit einer Präferenz für die Benutzerfreundlichkeit und Einfachheit} \label{noprivacy}
\textbf{Verarbeitete Daten zur Unterstützung der Softwareentwicklung in DevOps-Tools:} \newline
Zu Beginn der Interviews haben alle Befragten angegeben, mit persönlichen Daten in Berührung zu kommen und diese auch zu verarbeiten. Dabei haben alle vier Softwareentwickler angegeben, die Nutzeraktivitäten und den Programmcode von Kollegen 
einsehen und verarbeiten zu können. Dies erfolge zur Nachvollziehbarkeit der Builds und Deployments, des Programmcodes selbst, z.B. durch Git Commits und zur allgemeinen Fehlerbehebung bei fehlgeschlagenen Builds bzw. fehlerhaftem Programmcode. 
Zwei Befragte haben zudem angegeben, Zugriff auf sensible Kundendaten in Form von Klarnamen, Anschriften, E-Mail Adressen uvm. zu besitzen: Dies ist bei jenen der Fall, die häufig mit externen Ansprechpartnern und Unternehmen in Kontakt treten 
und mit diesen zusammenarbeiten müssen. \newline \newline
\textbf{Tatsächliche Aufbewahrungsdauer von Log-Dateien:} \newline
Da ein sehr großer Anteil der Befragten von über 85\% (vgl. Tabelle \ref{tab:generaldata}) angegeben haben, Log-Dateien aktiv zu nutzen, ist es für den weiteren Verlauf der Analyse erforderlich, die tatsächliche Aufbewahrungsdauer der genutzten Log-Dateien 
im Unternehmen zu spezifizieren. Hier haben alle Befragten angegeben, dass alle Log-Dateien in der Regel für eine unbegrenzte bzw. unbestimmte Dauer aufbewahrt werden. Die Stakeholder haben zudem meist keine Kenntnis über die genaue Aufbewahrungsdauer und 
beziehen sich bei ihrer Aussage auf die Existenz von sehr alten Log-Dateien (über 5 Jahre). Die Ausnahme dieser Regel betrifft manuell durchgeführte Bereinigungen von Speicher und eingeleitete Löschung spezifischer Log-Dateien: Bei einem Befragten kam es vor,
dass das System einen Absturz erlitten hat, welche die Log-Dateien unbenutzbar gemacht haben, wohingegen bei einem anderen der Serverspeicher geleert werden musste und dieses Vorgehen sämtliche Log-Dateien gelöscht hat. \newline Aus diesen Aussagen lässt sich 
ableiten, dass ein Vorgehen zur automatisierten Löschung von Log-Dateien bei den Befragten bisher keine Anwendung gefunden hat. \newline \newline
\textbf{Interesse an anderen persönlichen Daten in DevOps-Tools:} \\
\textbf{Gründe für das Interesse an Zeitstempeln in DevOps-Tools:} \\
\textbf{Interesse an persönlichen Daten in Review-Tools:} \\
\textbf{Interesse an persönlichen Daten in Administrationstools:} \\
\textbf{Detailgrad von Issues in Ticketverwaltungstools:} \\
\textbf{Personenbezug von Issue-Kommentaren in Ticketverwaltungstools:} \\
\textbf{Allgemeine Auswertungen von persönlichen Daten aus DevOps-Tools:} \\
\textbf{Relevanz von persönlichen Daten in Builds:} \\
\textbf{Interesse an Log-Dateien:} \\
\textbf{Mögliche Verarbeitungsinteressen von persönlichen Daten in anderen Unternehmen:} \\

\section{Kategorien mit unstimmigen Ergebnissen} \label{noclearresult}
\textbf{Detailgrad von Zeitstempeln in DevOps-Tools:} \\
\textbf{Sonstige Verarbeitungsinteressen von persönlichen Daten:} \\

\section{Vorgeschlagene Alternativen}

\section{Zusammenstellung der Ergebnisse}





	% !TeX program = lualatex
\chapter{Diskussion der Ergebnisse} % Main chapter title
\label{Discussion} % For referencing the chapter elsewhere, use \ref{Discussion}

\section{Implikationen in der Theorie}

\section{Implikationen in der Praxis}

\section{Limitation der Arbeit}
	% !TeX program = lualatex
\chapter{Fazit und Ausblick} % Main chapter title
\label{Conclusion} % For referencing the chapter elsewhere, use \ref{Conclusion}

Im Umfang dieser Analyse wurde die Privatsphäre in Verbindung mit Stakeholdern und DevOps in Softwareentwicklungsunternehmen präsentiert
und anhand Belege aktueller Forschungen ein erster Versuch einer praktischen Feldanalyse gewagt. Anhand der Expertenbefragungen konnten so einzelne
Kategorisierungen herausgearbeitet werden, welche entweder einer Präferenz zur Einfachheit und Benutzerfreundlichkeit oder der Privatsphäre im Arbeitsalltag
und -ablauf zugeordnet wurden. Die einzelnen Kategorien deckten sich dabei mit den Annahmen aus der Forschung. \newline
Da eine so geringe Anzahl an gewählten Experten für eine Analyse jedoch nicht als vollständig repräsentativ angesehen werden kann, ist diese Arbeit als eine
Unterstützung der aktuellen Forschung in Bezug auf die Privatsphäre der Stakeholder und deren Sichtweisen am Arbeitsplatz in Softwareentwicklungsunternehmen
zu betrachten. \newline \newline
Nun bleibt zu hoffen, dass zusätzlich zu den aktuellen Entwicklungen, welche mit der zunehmenden Digitalisierung Eintritt in den Alltag aller finden, neue Entdeckungen
auch in Hinsicht auf Privatsphäre in der Zukunft kritisch begutachtet werden. Hilfreich ist es allemal, wenn \enquote{Big Player} der Software- bzw. Hardwareentwicklung
ein Vorbild repräsentieren und die Relevanz der Privatsphäre immer wieder betonen. \newline Vielleicht überlegt es sich manch einer auf diese Weise zweimal, eine Kundenkarte
wie Payback, der DeutschlandCard o.Ä. zu beantragen und beim Einkaufen zu verwenden.

	% APPENDICES
	\addtocontents{toc}{\string\def\string\chaptername{Appendix}}
	\appendix
	\renewcommand{\thesection}{\thechapter.\arabic{section}}
	\renewcommand{\thesubsection}{\thesection.\arabic{subsection}}
	\renewcommand{\thesubsubsection}{\thesubsection.\arabic{subsubsection}}

	% Appendix A
\chapter{Verbundene Arbeiten}

\section{Stellenausschreibung zur Befragung von Softwareentwicklern bzw. -administratoren} \label{ausschreibung}
Im Rahmen einer Abschlussarbeit, welche sich mit der qualitativen Analyse von Stakeholdern in einem IT-Unternehmen zu Sichtweisen auf Privatsphäre befasst, sind wir auf der Suche nach potenziellen Befragungen von Softwareentwicklern/-administratoren. Sie sollten bereits
\begin{itemize}
    \item in einem Softwareentwicklungsunternehmen gearbeitet haben
    \item Erfahrungen mit der Arbeit im Team haben
    \item mit einem Versionsverwaltungs-Tool (z.B. Git, SVN etc.) gearbeitet haben
    \item grobe Kenntnisse über interne Abläufe besitzen
\end{itemize}
Melden Sie sich bei Interesse bitte bei mir, \textbf{Samet Akcabay}, unter der E-Mail-Adresse \textbf{samet-murat.akcabay@stud.uni-bamberg.de}!

\section{Datenschutzerklärung} \label{erklaerung}
\textbf{Datenschutzerklärung zur Befragung im Rahmen der Abschlussarbeit \enquote{Qualitative Analyse von Stakeholdern in einem Softwareentwicklungsunternehmen zu Sichtweisen auf Privatsphäre bei Softwareentwicklern bzw. -administratoren}} \newline \newline
Wir, der Lehrstuhl für Privatsphäre und Sicherheit in Informationssystemen, legen großen Wert auf Datenschutz und das Recht auf informationelle Selbstbestimmung. Aufgrund dessen möchten wir Sie mit dieser Datenschutzerklärung über die Erhebung, Speicherung und Verarbeitung der Daten, die im Zusammenhang mit der Abschlussarbeit „Qualitative Analyse von Stakeholdern in einem Softwareentwicklungsunternehmen zu Sichtweisen auf Privatsphäre bei Softwareentwicklern bzw. -administratoren“ erhoben werden, informieren. \newline \newline
Alle personenbezogenen Daten werden in keinster Weise betrachtet, verarbeitet oder an eine andere Instanz weitergeleitet. Bei der Befragung geht es lediglich darum, die Interessen in verschiedenen Unternehmen sowie von verschiedenen Interessensgruppen zu analysieren. Jede einzelne Angabe in jeder Befragung wird im Anschluss zu allgemeinen Punkten kategorisiert und zu Mustern zusammengefasst. Nach erfolgreicher Verarbeitung der nicht-personenbezogenen Daten werden sämtliche Aufnahmen und Transkriptionen unwiderruflich gelöscht. \newline \newline
Die befragten Personen willigen bei Teilnahme an der Befragung im Rahmen der Abschlussarbeit dieser Datenschutzerklärung zu. Sie haben jederzeit das Recht, Ihre Angaben zu widerrufen. In diesem Fall werden alle von Ihnen gespeicherten Daten unwiderruflich gelöscht und eine Verarbeitung dieser findet nicht statt. \newline \newline
Sie können jederzeit den Lehrstuhl für Privatsphäre und Sicherheit in Informationssystemen oder den Verfasser der Abschlussarbeit bei Fragen oder Anregungen kontaktieren: \newline \newline
\textbf{Universität Bamberg\newline
Lehrstuhl für Privatsphäre und Sicherheit in Informationssystemen An der Weberei 5\newline
96047 Bamberg\newline \newline
Raum:} WE5/05.064\newline
\textbf{Sekretariat:} WE5/05.063\newline 
\textbf{Telefon:} +49 (0)951–863 2661\newline \newline
\textbf{Samet Akcabay\newline
An der Spinnerei 11 Zimmer-Nr. D29 \newline96047 Bamberg\newline \newline
Telefon:} +49 (0)172-315 5296 \newline
\textbf{E-Mail:} samet-murat.akcabay@stud.uni-bamberg.de


\section{Online-Formular zum Ausfüllen über Microsoft Forms} \label{forms}
\begin{enumerate}
    \item Welche Rolle übernehmen Sie im Unternehmen?
    \item Administrieren Sie auch Server?
    \item Welche Daten verarbeiten Sie allgemein bei der Softwareentwicklung? Welche Daten schauen Sie sich explizit mithilfe von DevOps-Tools an, die Sie bei der Softwareentwicklung unterstützen?
    \item Wo liegen diese Daten, die von Ihnen verarbeitet werden? Finden sich unter diesen auch personenbezogene Daten?
    \item Was bedeutet für Sie allgemein "Daten verarbeiten"? Welche Abläufe bringen Sie damit in Verbindung?
    \item Wie genau können Sie diese von Ihnen genannten Daten einsehen?
    \item Schauen Sie sich bestimmte Daten explizit an bzw. legen Sie Wert darauf, dass bestimmte Daten gegeben sind?
    \item Schauen Sie sich auch explizit an, wer an welchen Issues/Builds (falls vorhanden) in welchem Projekt gearbeitet hat?
    \item Reviewen Sie auch Code von anderen Entwicklern? Falls ja, was genau schauen Sie sich beim verwendeten Review-Tool an (z.B. Name des Reviewers, Zeitstempel, ID etc.)? Worauf legen Sie besonders wert?
    \item Allgemein gefragt: Was wäre für Sie eine angemessene Zeit zur Aufbewahrung von personenbezogenen Daten? Bitte begründen Sie Ihre Antwort.
    \item Wie lange werden die von Ihnen genannten Daten bei Ihnen im Unternehmen tatsächlich aufbewahrt?
    \item Zurück zu Issues/Builds: Schauen Sie sich auch an, wann/ob/von wem der Issue erstellt/bearbeitet/kommentiert wurde?
    \item Welche Interessen verfolgen Sie generell, wenn Sie sich Zeitstempel in DevOps-Tools ansehen? Worauf legen Sie besonders wert?
    \item Von der anderen Seite aus betrachtet: Worauf könnten Sie verzichten?
    \item Was glauben Sie, zu welchem Zweck man diese Tools verwendet? (z.B. Nachvollziehbarkeit, Leistungsbeurteilung, Qualitätssicherung etc.)
    \item Könnten Sie sich andere Auswertungen aus den Daten vorstellen? (Auch, was Sie persönlich noch nicht selbst erlebt/gemacht haben, sondern welche im Interesse des Unternehmens liegen)
    \item Verwenden/arbeiten Sie mit Log-Dateien?
    \item Falls ja: Finden Sie es wichtig, dass Benutzername und Zeitstempel in den Log-Dateien vorhanden sind? Falls nicht, worauf würden Sie stattdessen setzen?
    \item Wie lange werden Log-Dateien bei Ihnen aufbewahrt? (z.B. von Tickets, Builds, Issues etc.)
    \item Fällt Ihnen noch etwas bisher nicht genanntes ein, worauf Sie wert legen und was Sie sich ansehen?
    \item Fallen Ihnen andere Verarbeitungsinteressen bzw. Auswertungsansätze ein, die andere Firmen haben könnten?

\end{enumerate}
	% Appendix B

\chapter{Interviewleitfaden} \label{interview}

% \begin{table*}[t]
%     \caption{Verwendeter Leitfaden für die Befragung zu den Sichtweisen auf die Privatsphäre in Softwareentwicklungsunternehmen (angelehnt an \cite{Helfferich:2011aa})}
%     \label{tab:leitfaden}
%     \centering
%     \small % use smaller fontsize in the table
%     {\renewcommand{\arraystretch}{2} % increase vertical space between rows
%     \begin{tabularx}{\linewidth}{@{}llllX@{}} % @{} omits outer horizontal margins, the "X" column uses up all remaining available space 
%       \toprule
%       Leitfrage & Check & Konkrete Fragen & Aufrechterhaltungs-/Steuerungsfragen \\
%       \midrule
%         \enquote{Welche Daten verarbeiten Sie?}                                                                             & Desinteresse                  & Persönliche Daten in DevOps-Tools             \\
%         \enquote{Was schauen Sie sich konkret an?}                                                                          & Nutzeraktivität               & Persönliche Daten in Administrationstools     \\
%         \enquote{In welcher Auflösung brauchen Sie den Zeitstempel?}                                                        & Desinteresse                  & Nutzung von Tools zur Administration          \\
%         \enquote{Gibt es etwas anderes, was Sie sich ansehen?}                                                              & Desinteresse                  & Persönliche Daten in DevOps-Tools             \\
%         \enquote{Schauen Sie sich auch an, wer spezifisch was gebaut hat?}                                                  & Benutzername                  & Persönliche Daten in DevOps-Tools             \\
%         \enquote{Was schauen Sie sich beim Review-Tool an?}                                                                 & Benutzername                  & Persönliche Daten in DevOps-Tools             \\
%         \enquote{Wie lange ist es für Sie wichtig, dass diese Daten aufbewahrt werden?}                                     & Benutzername                  & Persönliche Daten in DevOps-Tools             \\
%         \enquote{Wie lange werden diese tatsächlich aufbewahrt?}                                                            & Benutzername                  & Persönliche Daten in DevOps-Tools             \\
%         \enquote{Was für Daten schauen Sie sich in Issues an?}                                                              & Benutzername                  & Persönliche Daten in DevOps-Tools             \\
%         \enquote{Schauen Sie sich auch an, von wem ein Kommentar editiert wurde?}                                           & Benutzername                  & Persönliche Daten in DevOps-Tools             \\
%         \enquote{Welche Interessen verfolgen Sie generell bei Zeitstempeln in DevOps-Tools?}                                & Benutzername                  & Persönliche Daten in DevOps-Tools             \\
%         \enquote{Welche Auswertungen, die sie selbst nicht machen, können sie sich sonst mit diesen Daten vorstellen?}      & Benutzername                  & Persönliche Daten in DevOps-Tools             \\
%         \enquote{Schauen Sie sich auch Logs an?}                                                                            & Benutzername                  & Persönliche Daten in DevOps-Tools             \\
%         \enquote{Welche Auswertungen, die sie selbst nicht machen, können sie sich sonst mit diesen Daten vorstellen?}      & Benutzername                  & Persönliche Daten in DevOps-Tools             \\
%         \enquote{Ist es für Sie auch wichtig, wer wann was gemacht hat?}                                                    & Benutzername                  & Persönliche Daten in DevOps-Tools             \\
%         \enquote{Wissen Sie, wie lange diese Log-Dateien aufbewahrt werden?}                                                & Benutzername                  & Persönliche Daten in DevOps-Tools             \\
%         \enquote{Welche Tools verwenden Sie zur Administration?}                                                            & Benutzername                  & Persönliche Daten in DevOps-Tools             \\
%         \enquote{Haben Sie da die Möglichkeit einzusehen, wer wann was gemacht hat?}                                        & Benutzername                  & Persönliche Daten in DevOps-Tools             \\
%         \enquote{Warum ist das \enquote{wer} für Sie relevant?}                                                             & Benutzername                  & Persönliche Daten in DevOps-Tools             \\
%         \enquote{Können Sie sich andere Verarbeitungsinteressen vorstellen?}                                                & Benutzername                  & Persönliche Daten in DevOps-Tools             \\
%       \bottomrule
%     \end{tabularx}
%     }
%   \end{table*}
	%\include{appendices/appendixC}

	% BIBLIOGRAPHY
	\newgeometry{
		inner=2cm,
		outer=2cm, 
		marginparwidth=0cm,
		marginparsep=0mm,
		bindingoffset=.5cm,
		top=1.5cm,
		bottom=2.5cm,
		includehead,
		includefoot
	}
	\addchap{Literaturverzeichnis}

	\setlength\columnsep{2em}
	\begin{multicols}{2}
		\begin{refcontext}[sorting=nyt]
			\renewcommand*{\bibfont}{\small\RaggedRight}
			\linespread{1.0}\selectfont
			\printbibliography[heading=none]
		\end{refcontext}
	\end{multicols}

	% DECLARATION
	\begin{declaration}
		\addchaptertocentry{\authorshipname}
		Ich erkläre hiermit gemä\ss\ \S~17 Abs.\,2 APO, dass ich die vorstehende {\thesistype}arbeit selbständig\\ verfasst und keine anderen als die angegebenen Quellen und Hilfsmittel benutzt habe.\\
		\bigskip
		\bigskip
		\\
		\begin{tabular}{@{}l@{}}
  		Bamberg, den \rule[-0.8em]{10em}{0.5pt}\\[2ex]
  		~
		\end{tabular}
		\hspace{\fill}%
		\begin{tabular}{@{}c@{}}
  			\rule[-0.8em]{20em}{0.5pt}\\[2ex]
  			\authorname
			\end{tabular}\hspace{\fill}
	\end{declaration}
\end{document}