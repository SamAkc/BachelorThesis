% Appendix A
\chapter{Related Work}

\section{Stellenausschreibung zur Befragung von Softwareentwicklern bzw. -administratoren} \label{ausschreibung}
Im Rahmen einer Abschlussarbeit, welche sich mit der qualitativen Analyse von Stakeholdern in einem IT-Unternehmen zu Sichtweisen auf Privatsphäre befasst, sind wir auf der Suche nach potenziellen Befragungen von Softwareentwicklern/-administratoren. Sie sollten bereits
\begin{itemize}
    \item in einem Softwareentwicklungsunternehmen gearbeitet haben
    \item Erfahrungen mit der Arbeit im Team haben
    \item mit einem Versionsverwaltungs-Tool (z.B. Git, SVN etc.) gearbeitet haben
    \item grobe Kenntnisse über interne Abläufe besitzen
\end{itemize}
Melden Sie sich bei Interesse bitte bei mir, \textbf{Samet Akcabay}, unter der E-Mail-Adresse \textbf{samet-murat.akcabay@stud.uni-bamberg.de}!

\section{Datenschutzerklärung} \label{erklaerung}
\textbf{Datenschutzerklärung zur Befragung im Rahmen der Abschlussarbeit \enquote{Qualitative Analyse von Stakeholdern in einem Softwareentwicklungsunternehmen zu Sichtweisen auf Privatsphäre bei Softwareentwicklern bzw. -administratoren}} \newline \newline
Wir, der Lehrstuhl für Privatsphäre und Sicherheit in Informationssystemen, legen großen Wert auf Datenschutz und das Recht auf informationelle Selbstbestimmung. Aufgrund dessen möchten wir Sie mit dieser Datenschutzerklärung über die Erhebung, Speicherung und Verarbeitung der Daten, die im Zusammenhang mit der Abschlussarbeit „Qualitative Analyse von Stakeholdern in einem Softwareentwicklungsunternehmen zu Sichtweisen auf Privatsphäre bei Softwareentwicklern bzw. -administratoren“ erhoben werden, informieren. \newline \newline
Alle personenbezogenen Daten werden in keinster Weise betrachtet, verarbeitet oder an eine andere Instanz weitergeleitet. Bei der Befragung geht es lediglich darum, die Interessen in verschiedenen Unternehmen sowie von verschiedenen Interessensgruppen zu analysieren. Jede einzelne Angabe in jeder Befragung wird im Anschluss zu allgemeinen Punkten kategorisiert und zu Mustern zusammengefasst. Nach erfolgreicher Verarbeitung der nicht-personenbezogenen Daten werden sämtliche Aufnahmen und Transkriptionen unwiderruflich gelöscht. \newline \newline
Die befragten Personen willigen bei Teilnahme an der Befragung im Rahmen der Abschlussarbeit dieser Datenschutzerklärung zu. Sie haben jederzeit das Recht, Ihre Angaben zu widerrufen. In diesem Fall werden alle von Ihnen gespeicherten Daten unwiderruflich gelöscht und eine Verarbeitung dieser findet nicht statt. \newline \newline
Sie können jederzeit den Lehrstuhl für Privatsphäre und Sicherheit in Informationssystemen oder den Verfasser der Abschlussarbeit bei Fragen oder Anregungen kontaktieren: \newline \newline
\textbf{Universität Bamberg\newline
Lehrstuhl für Privatsphäre und Sicherheit in Informationssystemen An der Weberei 5\newline
96047 Bamberg\newline \newline
Raum:} WE5/05.064\newline
\textbf{Sekretariat:} WE5/05.063\newline 
\textbf{Telefon:} +49 (0)951–863 2661\newline \newline
\textbf{Samet Akcabay\newline
An der Spinnerei 11 Zimmer-Nr. D29 \newline96047 Bamberg\newline \newline
Telefon:} +49 (0)172-315 5296 \newline
\textbf{E-Mail:} samet-murat.akcabay@stud.uni-bamberg.de


\section{Online-Formular zum Ausfüllen über Microsoft Forms} \label{forms}
\begin{enumerate}
    \item Welche Rolle übernehmen Sie im Unternehmen?
    \item Administrieren Sie auch Server?
    \item Welche Daten verarbeiten Sie allgemein bei der Softwareentwicklung? Welche Daten schauen Sie sich explizit mithilfe von DevOps-Tools an, die Sie bei der Softwareentwicklung unterstützen?
    \item Wo liegen diese Daten, die von Ihnen verarbeitet werden? Finden sich unter diesen auch personenbezogene Daten?
    \item Was bedeutet für Sie allgemein "Daten verarbeiten"? Welche Abläufe bringen Sie damit in Verbindung?
    \item Wie genau können Sie diese von Ihnen genannten Daten einsehen?
    \item Schauen Sie sich bestimmte Daten explizit an bzw. legen Sie Wert darauf, dass bestimmte Daten gegeben sind?
    \item Schauen Sie sich auch explizit an, wer an welchen Issues/Builds (falls vorhanden) in welchem Projekt gearbeitet hat?
    \item Reviewen Sie auch Code von anderen Entwicklern? Falls ja, was genau schauen Sie sich beim verwendeten Review-Tool an (z.B. Name des Reviewers, Zeitstempel, ID etc.)? Worauf legen Sie besonders wert?
    \item Allgemein gefragt: Was wäre für Sie eine angemessene Zeit zur Aufbewahrung von personenbezogenen Daten? Bitte begründen Sie Ihre Antwort.
    \item Wie lange werden die von Ihnen genannten Daten bei Ihnen im Unternehmen tatsächlich aufbewahrt?
    \item Zurück zu Issues/Builds: Schauen Sie sich auch an, wann/ob/von wem der Issue erstellt/bearbeitet/kommentiert wurde?
    \item Welche Interessen verfolgen Sie generell, wenn Sie sich Zeitstempel in DevOps-Tools ansehen? Worauf legen Sie besonders wert?
    \item Von der anderen Seite aus betrachtet: Worauf könnten Sie verzichten?
    \item Was glauben Sie, zu welchem Zweck man diese Tools verwendet? (z.B. Nachvollziehbarkeit, Leistungsbeurteilung, Qualitätssicherung etc.)
    \item Könnten Sie sich andere Auswertungen aus den Daten vorstellen? (Auch, was Sie persönlich noch nicht selbst erlebt/gemacht haben, sondern welche im Interesse des Unternehmens liegen)
    \item Verwenden/arbeiten Sie mit Log-Dateien?
    \item Falls ja: Finden Sie es wichtig, dass Benutzername und Zeitstempel in den Log-Dateien vorhanden sind? Falls nicht, worauf würden Sie stattdessen setzen?
    \item Wie lange werden Log-Dateien bei Ihnen aufbewahrt? (z.B. von Tickets, Builds, Issues etc.)
    \item Fällt Ihnen noch etwas bisher nicht genanntes ein, worauf Sie wert legen und was Sie sich ansehen?
    \item Fallen Ihnen andere Verarbeitungsinteressen bzw. Auswertungsansätze ein, die andere Firmen haben könnten?

\end{enumerate}