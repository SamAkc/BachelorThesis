% !TeX program = lualatex
\chapter{Diskussion der Ergebnisse} % Main chapter title
\label{Discussion} % For referencing the chapter elsewhere, use \ref{Discussion}
Der Hintergrund dieser Arbeit lässt sich damit begründen, dass die Relevanz der Privatsphäre im 21. Jahrhundert aufgezeigt und durch gewählte Kategorisierungen der Aussagen aus den
Experteninterviews untermauert wird. \newline
Die aufgezeigten Ergebnisse und die folgenden Diskussionen in dieser Analyse dienen dem Zweck, den aktuellen Kenntnisstand zu unterstützen und zukünftige, potenzielle Analysen zu diesem Thema zu motivieren. Hierzu werden die aufgearbeiteten Ergebnisse 
mit bestehender Literatur korreliert und für den Einsatz in der Praxis ausgewertet. Im Anschluss werden die Einschränkungen, welcher diese Arbeit unterlag, erläutert.

\section{Implikationen für bestehende Theorie}
Anhand der absoluten Zahl der entsprechenden Kategorien aus Kapitel \ref{Results}, welche der Benutzerfreundlichkeit und damit dem Vernachlässigen der
Privatsphäre zugeordnet werden können, kann die Schlussfolgerung mithilfe der Aussagen der Stakeholder gefällt werden, dass aus Sicht der Stakeholder die Benutzerfreundlichkeit bzw. Einfachheit
im Arbeitsalltag für den Großteil der Stakeholder wichtiger als die Privatsphäre ist. Begründet wird dies häufig dadurch, dass sich diverse Art und Weisen seiner eigenen Arbeit, in einem Softwareentwicklungsunternehmen,
nachzugehen, entweder etabliert hat (besonders in Abschnitt \ref{comments} deutlich), eine Ungewissheit seitens der Stakeholder (vgl. Abschnitt \ref{logs}) vorherrscht oder eine grundlegende Funktion durch die Anonymisierung
von persönlichen Daten nicht mehr gewährleistet wäre (vgl. Sinnbeeinträchtigung durch editierte Kommentare in der Kommentarsektion von Tickets, Abschnitt \ref{comments}). \newline
Diese Behauptung wird durch das zu Anfang erwähnte Phänomen des \enquote{Privacy Paradox} unterstützt: Auf der einen Seite wird die Vorstellung angegeben, dass beispielsweise eine Auswertung der eigenen Arbeitsqualität 
durch persönliche Daten (z.B. durch Benutzernamen und Zeitstempel in Programmcode, Builds etc.) in anderen Softwareentwicklungsunternehmen stattfinden könnte und eine allgemeine Ablehnung gegen solche Methoden vorherrscht. Sichtbar ist dies
an der Aussage eines Befragten, welcher behauptet, dass \enquote{man [...] von den Commitzeiten [nicht] darauf schließen [kann], wann jemand gearbeitet hat. Manche Menschen leiten das vielleicht so ab, aber das ist komplett falsch}\sidenotetext{Aussage eines
Befragten dieser qualitativen Analyse}. Verstärkt wird diese Aussage durch die Tatsache, dass vor allem Verbraucher in Deutschland im Vergleich zu anderen Ländern überdurchschnittlich auf ihre eigene Privatsphäre achten \cite{IfH:2015aa} und mit einer steigenden Tendenz
Bedenken in Bezug auf diese äußern \cite{Bansal:2016aa}. \newline
Auf der anderen Seite möchte man aber seiner eigenen Arbeit so gut wie möglich nachgehen. Dies wird insofern deutlich, dass die Befragten beispielsweise angeben, bei einem Fehler die Möglichkeit zu haben, möglichst viele Details (und persönliche Daten) abrufen zu können (vgl. Bevorzugter Detailgrad von Zeitstempeln
in DevOps Tools aus Kapitel \ref{Results}). \newline
Grundlegend kann also der Entschluss gefasst werden, dass der bisherige Forschungsstand begründet und damit, die Kenntnisse und Befürchtungen in Bezug auf die Erfassung und Auswertung von persönlichen Daten in Softwareentwicklungsunternehmen berechtigt ist: Wie beispielsweise aus den Kategorien \enquote{Verarbeitete Daten zur Unterstützung
der Softwareentwicklung in DevOps-Tools}, \enquote{Bevorzugter Detailgrad von Zeitstempeln} und \enquote{Allgemeine Auswertungen von persönlichen Daten aus DevOps-Tools} deutlich wurde, besteht aus der Sicht der Stakeholder ein großes Interesse an der Erfassung von persönlichen Daten (hier: in Form
von Benutzernamen und Zeitstempel). Zusätzlich wurde der unmögliche Verzicht auf diese in der Kategorie \enquote{Interesse an Benutzernamen und Zeitstempeln in editierten Issue-Kommentaren und -Beschreibungen in Ticketverwaltungstools} angegeben - einem Sektor, welcher keinen Einfluss auf die grundlegende Funktion des eigenen Arbeitsflusses bietet
und lediglich zur Einfachheit und Nachvollziehbarkeit von Inhalten Dritter existiert. Aus der letzten Kategorie \enquote{Mögliche Verarbeitungsinteressen von persönlichen Daten in an-deren Unternehmen} wird eine Skepsis und z.T. eine pessimistische Einstellung gegenüber anderen Unternehmen geäußert, welche mit ethisch fragwürdigen Methoden (Leistungsbeurteilung, Nachverfolgung,
Überprüfung ohne Kenntnis, z.B. während einer Bewerbung) agieren sollen. Diese genannten Punkte lassen sich nämlich den zuvor aus Kapitel \ref{personaldata} genannten Punkten (Existenz von Tools, welche die Erfassung und Auswertung von persönlichen Daten unterstützen; Bereitschaft vorhanden; im Interesse des Unternehmens).

\section{Implikationen für die Praxis}
Auf Grundlage der bisher durchgeführten Analyse des Forschungsstandes und der Expertenbefragungen, können im nächsten Schritt Schlussfolgerungen in Hinblick auf die Praxis für die Zukunft getroffen werden. Auf der einen Seite wird aus den herausgearbeiteten Kategorien zu den jeweiligen Sichtweisen der Stakeholder deutlich, dass die Einfachheit und Benutzerfreundlichkeit
im Arbeitsalltag stärker ins Gewicht fällt als die Privatsphäre. Dadurch lässt sich auf eine allgemeine Zufriedenheit und Akzeptanz mit dem aktuellen Umgang der eigenen, aber auch fremden (von Kollegen, Vorgesetzten, Externen) persönlichen Daten schließen. Zu lediglich zwei Kategorien (Detailgrad des Zeitstempels in DevOps-Tools und der tatsächlichen Aufbewahrungsdauer von Log-Dateien)
Vorschläge zu alternativen Methoden erläutert. Trotz zunehmender Relevanz der Privatsphäre im Allgemeinen, scheint ein Großteil, weiterhin den Fokus auf die Einfachheit und Benutzerfreundlichkeit zu legen - zumindest am Arbeitsplatz. \newline
Diese vorgeschlagenen Alternativen können aber erste Schritte darstellen, in welchen Aspekten persönliche Daten anonymisiert bzw. komplett entfernt werden können. Weiterhin kann aber auch das Miteinbeziehen der jeweiligen Sichtweisen der Stakeholder in einzelnen Softwareentwicklungsunternehmen ein grundlegendes Konzept zum Umgang mit persönlichen Daten darstellen: Wie aus den Ergebnissen dieser Analyse und dem aktuellen Forschungsstand herausgelesen
werden kann, können die einzelnen Ansichtsweisen variieren und eine gute Möglichkeit, den bestmöglichen Umgang mit persönlichen Daten zu gewährleisten, kann durch die Kommunikation untereinander erzielt werden kann. Dabei können Leitfragen wie \enquote{Wie weit sind die einzelnen Stakeholder bereit, ihre eigenen (persönlichen) Daten im Sinne der kooperativen Arbeit mit Kollegen offenzulegen?} und \enquote{Welche Funktionen sind für die
einzelnen Stakeholder weniger relevant als die eigene Privatsphäre und können daher durch eine selbstbestimmte Alternative ersetzt werden?} beim ersten Aufbau einer Arbeitsumgebung oder dem nächsten Meeting Einsatz finden.

\section{Limitation der Arbeit}
Im Laufe der Aufarbeitung dieser qualitativen Analyse unterliegt die Arbeit diversen Limitationen. Aufgrund der subjektiven Natur der Sichtweisen der einzelnen Stakeholder und der Anzahl von insgesamt sieben Befragten stellt diese Analyse eine Unterstützung zur bestehenden Literatur dar und kann durch mehrmalige oder veränderte Durchführungen von
Expertenbefragungen unterschiedliche Ergebnisse erzielen. Zudem befinden sich im gewählten Zeitraum zur Durchführung der einzelnen Interviews potenzielle Interessenten in der Situation, sich aufgrund der COVID-19-Pandemie im Home-Office zurechtfinden zu müssen und in der Regel keine Zeit bzw. Motivation für externe Anfragen (wie z.B. für Interviews)
aufbringen zu können. Durch einen längeren Zeitraum zur Durchführung und der Steigerung der Anzahl der Befragungen können genauere und repräsentativere Ergebnisse in der Zukunft erzielt werden. 