% !TeX program = lualatex
\chapter{Diskussion der Ergebnisse} % Main chapter title
\label{Discussion} % For referencing the chapter elsewhere, use \ref{Discussion}
Der Hintergrund dieser Arbeit lässt sich damit begründen, dass die Relevanz der Privatsphäre im 21. Jahrhundert aufgezeigt und durch gewählte Kategorisierungen der Aussagen aus den
Experteninterviews untermauert wird. Anhand der absoluten Zahl der entsprechenden Kategorien aus Kapitel \ref{Results}, welche der Benutzerfreundlichkeit und damit dem Vernachlässigen der
Privatsphäre zugeordnet werden können, kann die Schlussfolgerung mithilfe der Aussagen der Stakeholder gefällt werden, dass aus Sicht der Stakeholder die Benutzerfreundlichkeit bzw. Einfachheit
im Arbeitsalltag für den Großteil der Stakeholder wichtiger als die Privatsphäre ist. Begründet wird dies häufig dadurch, dass sich diverse Art und Weisen seiner eigenen Arbeit, in einem Softwareentwicklungsunternehmen,
nachzugehen, entweder etabliert hat (besonders in Abschnitt \ref{comments} deutlich), eine Ungewissheit seitens der Stakeholder (vgl. Abschnitt \ref{logs}) vorherrscht oder eine grundlegende Funktion durch die Anonymisierung
von persönlichen Daten nicht mehr gewährleistet wäre (vgl. Sinnbeeinträchtigung durch editierte Kommentare in der Kommentarsektion von Tickets, Abschnitt \ref{comments}). \newline
Diese Behauptung wird durch das zu Anfang erwähnte Phänomen des \enquote{Privacy Paradox} unterstützt: Auf der einen Seite wird die Vorstellung angegeben, dass beispielsweise eine Auswertung der eigenen Arbeitsqualität 
durch persönliche Daten (z.B. durch Benutzernamen und Zeitstempel in Programmcode, Builds etc.) in anderen Softwareentwicklungsunternehmen stattfinden könnte und eine allgemeine Ablehnung gegen solche Methoden vorherrscht. Sichtbar ist dies
an der Aussage eines Befragten, welcher behauptet, dass \enquote{man [...] von den Commitzeiten [nicht] darauf schließen [kann], wann jemand gearbeitet hat. Manche Menschen leiten das vielleicht so ab, aber das ist komplett falsch}\sidenotetext{Aussage eines
Befragten dieser qualitativen Analyse}. Verstärkt wird diese Aussage durch die Tatsache, dass vor allem Verbraucher in Deutschland im Vergleich zu anderen Ländern überdurchschnittlich auf ihre eigene Privatsphäre achten \cite{IfH:2015aa} und mit einer steigenden Tendenz
Bedenken in Bezug auf diese äußern \cite{Bansal:2016aa}. \newline
Auf der anderen Seite möchte man aber seiner eigenen Arbeit so gut wie möglich nachgehen. Dies wird insofern deutlich, dass die Befragten beispielsweise angeben, bei einem Fehler die Möglichkeit zu haben, möglichst viele Details (und persönliche Daten) abrufen zu können (vgl. Bevorzugter Detailgrad von Zeitstempeln
in DevOps Tools aus Kapitel \ref{Results}). \newline \newline
In Hinblick auf die präsentieren Ergebnisse haben sich Implikationen in Bezug auf die Theorie und Praxis ergeben, welche in den folgenden Unterkapiteln näher betrachtet werden. Im Anschluss werden die Limitationen dieser qualitativen Analyse erläutert.

\section{Implikationen in der Theorie}

\section{Implikationen in der Praxis}

\section{Limitation der Arbeit}