% !TeX program = lualatex
\chapter{Fazit und Ausblick} % Main chapter title
\label{Conclusion} % For referencing the chapter elsewhere, use \ref{Conclusion}

Im Umfang dieser Analyse wurde die Privatsphäre in Verbindung mit Stakeholdern und DevOps in Softwareentwicklungsunternehmen präsentiert
und anhand Belege aktueller Forschung ein erster Versuch einer praktischen Feldanalyse gewagt. Anhand der Expertenbefragungen konnten so einzelne
Kategorisierungen herausgearbeitet werden, welche entweder einer Präferenz zur Einfachheit und Benutzerfreundlichkeit oder der Privatsphäre im Arbeitstag
und -ablauf zugeordnet wurden. Die einzelnen Kategorien deckten sich dabei mit den Annahmen aus der Forschung. \newline
Da eine so geringe Anzahl an gewählten Experten für eine Analyse jedoch nicht als vollständig repräsentativ angesehen werden kann, ist diese Arbeit als eine
Unterstützung der aktuellen Forschung in Bezug auf die Privatsphäre der Stakeholder und deren Sichtweisen am Arbeitsplatz in Softwareentwicklungsunternehmen
zu betrachten. \newline \newline
Nun bleibt zu hoffen, dass zusätzlich zu den aktuellen Entwicklungen, welche mit der zunehmenden Digitalisierung Eintritt in den Alltag aller finden, neue Entdeckungen
auch in Hinsicht auf Privatsphäre in der Zukunft kritisch begutachtet werden. Hilfreich ist es allemal, wenn \enquote{Big Player} der Software- bzw. Hardwareentwicklung
ein Vorbild repräsentieren und die Relevanz der Privatsphäre immer wieder betonen. \newline Vielleicht überlegt es sich manch einer auf diese Weise zweimal, eine Kundenkarte
wie Payback, der DeutschlandCard o.Ä. zu beantragen und beim Einkaufen zu verwenden.