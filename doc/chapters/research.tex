% !TeX program = lualatex
\chapter{Forschungsmethode} % Main chapter title
\label{Research} % For referencing the chapter elsewhere, use \ref{Research}

Dabei wurde die Vorgehensweise zur Durchführung von Experteninterviews gewählt: Hierzu wurden einzelne Softwareentwickler von diversen Unternehmen sorgfältig
anhand ihrer Qualifikation, Berufserfahrung und der Position im Unternehmen gewählt - um ein möglichst weites Spektrum an unterschiedlichen Personengruppen abzudecken, wurden diese in Berufseinsteiger,
welche kaum bis relativ wenig Berufserfahrung vorweisen können, mehrjährig Festangestellte Softwareentwickler und Team-Leads bzw. Projektleiter, welche zusätzlich noch verantwortlich für Entwicklergruppen sind,
untergliedert.

\section{Definition und Identifikation der Stakeholder in einem Softwareentwicklungsunternehmen}
Ziel dieser qualitativen Analyse ist es, die Denk- bzw. Sichtweisen von Stakeholdern in Softwareentwicklungsunternehmen zu erfassen und anhand dieser Ergebnisse, Schlüsse in Bezug auf die
eigene Privatsphäre dieser zu ziehen. Um ein grundlegendes Verständnis in den folgenden Kapiteln gewährleisten zu können ist es zunächst notwendig, diese Stakeholder zu definieren und anhand davon, 
potenzielle Stakeholder im Rahmen dieser Forschung zu identifizieren. \\ Als Stakeholder werden jene \enquote{Personen, Gruppen oder Institutionen bezeichnet, die von den Aktivitäten eines Unternehmens 
direkt oder indirekt betroffen sind oder [...] ein Interesse an diesen [...] haben} \cite{Fle:16} - dies kann von Kunden und Lieferanten bis zu den eigenen Mitarbeitern und Eigentümern reichen \cite{Fle:16}. 
Im Falle dieser Analyse wird die Sichtweise der einzelnen Stakeholder auf die eigene Privatsphäre im angestellten Unternehmen betrachtet. \\ Gemessen wird dieser Aspekt durch die Relevanz, wie viel Wert die befragten
Softwareentwickler beispielsweise darauf legen, personenbezogene Daten in einem möglichst weiten Spektrum für einen möglichst langen Zeitraum einsehen zu können und auf der Kehrseite, wie weit sie bereit sind, eigene
personenbezogene Daten preiszugeben bzw. mit Kollegen und anderen Stakeholdern zu teilen, um ein gut funktionierendes Glied eines Teams oder eines Unternehmens sein zu können.

In den folgenden Unterkapiteln werden die untergliederten Interessensgruppen und ihre entsprechende Notwendigkeit näher erläutert. Die Begründung für die spezifische Wahl dieser besagten Gruppen wird im
nächsten Kapitel klassifiziert.

\subsection{Berufseinsteiger, Praktikanten und Werkstudenten}
Die primäre Gruppe zeichnet sich durch Softwareentwickler aus, welche kaum bis wenig Berufserfahrung besitzen und eventuell geringere Qualifikationen, als die anderen Interessensgruppen aufweisen könnten.
Diese Merkmale finden sich in Berufseinsteigern wieder, welche vor Kurzem aus einer abgeschlossenen Ausbildung oder eines abgeschlossenen Studiums stammen und nun im Beruf Softwareentwickler sind. Selbe Punkte
lassen sich allerdings auch auf (längerfristige) Praktikanten und Werkstudenten, welche neben ihrem Studium in einem Unternehmen bis zu 20 Stunden in der Woche tätig sind, übertragen - es ist also wenig Berufserfahrung
vorhanden, aber grundlegende Kenntnisse in der Informatik, vor allem in Bezug auf DevOps, welche im nächsten Unterkapitel näher erläutert werden, lassen sich in diesen Interessensgruppen wiederfinden.

\subsection{Softwareentwickler}

\subsection{Team-Leads, Projektleiter (und Geschäftsführer?)}

\subsection{Externe Ansprechpartner}

\section{DevOps und deren Einsatz in Softwareentwicklungsunternehmen}

\subsection{Interne Kommunikationstools}

\subsection{Projektmanagement- und Wiki-Software}

\subsection{Git}

\section{Aktueller Stand und Technologien?}