% !TeX program = lualatex
\chapter{Forschungsmethode} % Main chapter title
\label{Research} % For referencing the chapter elsewhere, use \ref{Research}

Aufgrund der Natur der offenen Befragung wurde hierbei auf Leitfadeninterviews zurückgegriffen. Dadurch wurde gewährleistet, dass das Forschungsthema im Fokus der Befragung bleibt und Thematiken, welche von 
den Interviewpartnern angesprochen wurden, näher spezifiziert und erläutert werden konnten. Hier wurden insgesamt sieben Interviews mit Personen durchgeführt, welche aus zum Teil verschiedenen Bereichen und
Positionen in Softwareentwicklungsunternehmen stammen, aber dennoch mit DevOps und der Softwareentwicklung in Berührung stehen. \newline \newline
Besonders wichtig hierbei war es, soweit möglich, unterschiedliche Experten aus unterschiedlichen Bereichen und Positionen eines Unternehmens zu finden. Dabei waren alle Teilnehmer zum Zeitpunkt der Befragung 
über 23 Jahre alt. Fünf der sieben Interviews wurden hierbei bereits vor Beginn dieser Analyse vom Lehrstuhl für Privatsphäre und Sicherheit in Informationssystemen (kurz: PSI) durchgeführt und transkribiert, 
weswegen diese weiterverwendet und codiert werden konnten. Zwei der insgesamt sieben Interviews wurden während der Analyse durchgeführt, wovon eine persönlich und durch Audio-Aufzeichnung unterstützt und die 
Letzte durch ein zu ausfüllendes Online-Formular stattgefunden hat, da ein persönliches Gespräch aufgrund der COVID-19 Pandemie nicht möglich war. Einen Einfluss auf die Ergebnisse sollten diese Unterschiede 
in der Demographie und in der Durchführung jedoch nicht haben. \newline Der Leitfaden wurde vorab vom PSI-Lehrstuhl zur Verfügung gestellt und für diese Analyse weiterverwendet. Dieser ist im Anhang einsehbar 
(s. Anlage [...]).

\section{Vorbereitung}
Bevor die Experteninterviews durchgeführt wurden, mussten Vorarbeiten geleistet werden. Zunächst war es wichtig, eine Datenschutzerklärung zu erstellen, welche die zu interviewenden Personen über den Umgang mit
ihren persönlichen Daten aufgeklärt hat. Dadurch konnte zunächst gewährleistet werden, dass es sich bei dieser Forschung um ein seriöses Vorgehen handelt. Des weiteren konnte so den Interviewpartnern eine rechtliche
Grundlage über ihren Datenschutz und der persönlichen Daten zugesichert werden. Außerdem wurde angegeben, mit der Teilnahme an der Befragung der Datenschutzerklärung automatisch zuzustimmen. \newline 
Parallel dazu wurde eine Ausschreibung erstellt, welche das Interview beworben hat und in diversen sozialen Netzwerken (Twitter, LinkedIn, XING etc.) geteilt werden konnte. Hier wurde angegeben, Personen im Rahmen dieser 
Analyse zu suchen, welche bereits in einem Softwareentwicklungsunternehmen tätig waren oder tätig sind, Erfahrungen mit der Arbeit im Team und mit DevOps besitzen und grobe Kenntnisse über interne Abläufe (in Bezug auf 
Verwaltung des Unternehmens, Abläufe im Team, aber auch mit DevOps) besitzen. \newline \newline
Zu Personen, welche auf die Ausschreibung reagiert haben, wurde über E-Mail Kontakt aufgenommen und ein Termin zum gemeinsamen Treffen vereinbart. Die Befragung wurde in einem Seminarsaal durchgeführt und währenddessen mit
einem Notebook aufgezeichnet. Da eine persönliche Befragung nach Ausbruch der COVID-19 Pandemie nicht mehr möglich war, wurde mithilfe von \enquote{Microsoft Forms} ein Online-Formular mit den Leitfragen erstellt und als Direktlink per E-Mail
an die Interessenten weitergeleitet. Hierbei war es besonders wichtig, die Leitfragen so zu wählen, dass ein grundlegendes Verständnis, der Sinn und die Zusammenhänge dieser gewährleistet waren, da eine nachträgliche Erläuterung
und Spezifizierung aufgrund der gegebenen Umstände nicht mehr möglich waren.

\section{Durchführung und Transkription der Experteninterviews}


\section{Codierung der Ergebnisse}