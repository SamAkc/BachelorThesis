% !TeX program = lualatex
\chapter{Forschungsmethode} % Main chapter title
\label{Research} % For referencing the chapter elsewhere, use \ref{Research}

Aufgrund der Natur der offenen Befragung wurde hierbei auf Leitfadeninterviews zurückgegriffen. Dadurch wurde gewährleistet, dass das Forschungsthema im Fokus der Befragung bleibt und Thematiken, welche von 
den Interviewpartnern angesprochen wurden, näher spezifiziert und erläutert werden konnten. Hier wurden insgesamt sieben Interviews mit Personen durchgeführt, welche aus zum Teil verschiedenen Bereichen und
Positionen in Softwareentwicklungsunternehmen stammen, aber dennoch mit DevOps und der Softwareentwicklung in Berührung stehen. \newline \newline
Besonders wichtig hierbei war es, soweit möglich, unterschiedliche Experten aus unterschiedlichen Bereichen und Positionen eines Unternehmens zu finden. Dabei waren alle Teilnehmer zum Zeitpunkt der Befragung 
über 23 Jahre alt. Fünf der sieben Interviews wurden hierbei bereits vor Beginn dieser Analyse vom Lehrstuhl für Privatsphäre und Sicherheit in Informationssystemen (kurz: PSI) durchgeführt und transkribiert, 
weswegen diese weiterverwendet und codiert werden konnten. Zwei der insgesamt sieben Interviews wurden während der Analyse durchgeführt, wovon eine persönlich und durch Audio-Aufzeichnung unterstützt und die 
Letzte durch ein zu ausfüllendes Online-Formular stattgefunden hat, da ein persönliches Gespräch aufgrund der COVID-19 Pandemie nicht möglich war. Einen Einfluss auf die Ergebnisse sollten diese Unterschiede 
in der Demographie und in der Durchführung jedoch nicht haben. \newline Der Leitfaden wurde vorab vom PSI-Lehrstuhl zur Verfügung gestellt und für diese Analyse weiterverwendet. Dieser ist im Anhang einsehbar 
(s. Anlage [...]).

\section{Vorbereitung}
Bevor die Experteninterviews durchgeführt wurden, mussten Vorarbeiten geleistet werden. Zunächst war es wichtig, eine Datenschutzerklärung zu erstellen, welche die zu interviewenden Personen über den Umgang mit
ihren persönlichen Daten aufgeklärt hat. Dadurch konnte zunächst gewährleistet werden, dass es sich bei dieser Forschung um ein seriöses Vorgehen handelt. Des weiteren konnte so den Interviewpartnern eine rechtliche
Grundlage über ihren Datenschutz und der persönlichen Daten zugesichert werden. Außerdem wurde angegeben, mit der Teilnahme an der Befragung der Datenschutzerklärung automatisch zuzustimmen. \newline 
Parallel dazu wurde eine Ausschreibung erstellt, welche das Interview beworben hat und in diversen sozialen Netzwerken (Twitter, LinkedIn, XING etc.) geteilt werden konnte. Hier wurde angegeben, Personen im Rahmen dieser 
Analyse zu suchen, welche bereits in einem Softwareentwicklungsunternehmen tätig waren oder tätig sind, Erfahrungen mit der Arbeit im Team und mit DevOps besitzen und grobe Kenntnisse über interne Abläufe (in Bezug auf 
Verwaltung des Unternehmens, Abläufe im Team, aber auch mit DevOps) besitzen. \newline \newline
Zu Personen, welche auf die Ausschreibung reagiert haben, wurde über E-Mail Kontakt aufgenommen und ein Termin zum gemeinsamen Treffen vereinbart. Die Befragung wurde in einem Seminarsaal durchgeführt und währenddessen mit
einem Notebook aufgezeichnet. Da eine persönliche Befragung nach Ausbruch der COVID-19 Pandemie nicht mehr möglich war, wurde mithilfe von \enquote{Microsoft Forms} ein Online-Formular mit den Leitfragen erstellt und 
als Direktlink per E-Mail an die Interessenten weitergeleitet. Hierbei war es besonders wichtig, die Leitfragen so zu wählen, dass ein grundlegendes Verständnis, der Sinn und die Zusammenhänge dieser gewährleistet waren, 
da eine nachträgliche Erläuterung und Spezifizierung aufgrund der gegebenen Umstände nicht mehr möglich waren.

\section{Durchführung und Transkription der Experteninterviews}
Zu Beginn des Interviews wurde der zu Interviewende begrüßt und das Einverständnis zur Aufnahme des Gesprächs eingeholt. Danach wurde die Aufnahme gestartet und die Bestätigung zur Kenntnisnahme der Datenschutzerklärung 
sowie der Vermerk, diesem jederzeit widersprechen zu können erläutert. Da in der Stellenausschreibung und im darauffolgenden in Kontakt treten kommuniziert wurde, DevOps und diverse andere Arbeitsabläufe zu kennen, wurde 
darauf verzichtet, im Vorfeld Erläuterungen und Definitionen zu Begriffen wie DevOps und den Tools, Stakeholder oder Vergleichbares zu klären. Dieses Vorgehen wurde durch den durchweg flüssigen Dialog zusätzlich bestätigt.
Im Interview selbst wurde sich strikt an den Leitfaden gehalten: Sofern eine Abweichung der Antworten und somit eine Richtungsänderung stattgefunden hat, wurde, passend zum vorgegeben Leitfaden, eine \enquote{Alternativroute}
der Fragestellungen eingeschlagen. \newline \newline
Zur Abgrenzung der verschiedenen Stakeholder war es zu Anfang wichtig, welche Position im Unternehmen repräsentiert wurde und ob diese bereits oder gegenwärtig mit der Administration
von Servern beauftragt waren bzw. sind, da auf diese Art gezielt Fragen in Bezug auf diese Merkmale gestellt werden konnten (s. Tabelle \ref{tab:generaldata}). In den Befragungen stand im Vordergrund, die Sichtweise auf die Verarbeitung von eigenen und fremden
persönlichen Daten einzuholen. Falls verweigert wurde, eigene persönliche Daten (vor allem in DevOps) preiszugeben, wurden Alternativen angeregt. Zudem sollten die Experten dazu angeregt werden, ihre eigene Meinung und Denkweise
zu präsentieren, wie der Umgang mit persönlichen Daten in anderen Unternehmen bzw. unabhängig von der eigenen Person stattfindet. Bestärkt wurde dieser Aspekt durch die spezifische Frage nach Nutzung von Log-Dateien: Die Intention
hierbei war dabei, die Experten indirekt zu fragen, ob diese zunächst Einsicht in persönliche Daten von Kollegen durch Logs haben und ob sie diese Möglichkeit auch tatsächlich aktiv nutzen. 
\newline \newline
Die Befragungen fanden im Zeitraum zwischen März und Juli 2020 statt, wobei als spätmöglichster Tag zur Durchführung einer Befragung als der 15. Juli 2020 festgelegt wurde. Diese dauerten im Durchschnitt zehn Minuten. 
Im Anschluss erfolgte eine Transkription der Interviews. Dieses Prinzip wird auch die \enquote{fallbezogene Auswertung} \cite{Doring:2014aa} genannt - hierbei wird das qualitative Datenmaterial sequenziell, also von Anfang bis
Ende, nach Fällen ausgewertet und im Anschluss kodiert \cite{Doring:2014aa}.

Hierzu wurden die Aufnahmen wiederholt und die Fragestellungen mit den entsprechenden Antworten mittels einer Markdown-Datei festgehalten. Dieses Vorgehen wurde bei der Befragung in Form einer Umfrage redundant, da dieses als 
PDF-Dokument exportiert werden konnte. Zur fristgerechten Fertigstellung der Analyse und der gleichzeitigen Behinderung durch die COVID-19 Pandemie wurde sich nach Absprache mit der Leitung des PSI-Lehrstuhls darauf geeinigt, 
die Anzahl der Befragungen auf sieben zu limitieren.
\begin{table}
    \caption{Position und Erfahrung der Stakeholder im Unternehmen.} \label{tab:generaldata}
    \footnotesize
    \centering
    \begin{tabular}{r r r}
    \toprule
    & \multicolumn{2}{c}{\tabhead{Beobachtete Ergebnisse}} \\ \cmidrule(lr){2-3} \tabhead{Rolle} & \tabhead{Administration von Servern} & \tabhead{Nutzung von Log-Dateien} \\ \midrule
    Softwareentwickler&25\% Ja, 75\% Nein&Ja (Server- und Applikationslogs)\\
    Softwareentwickler und Berater&Nein&Keine Angabe\\
    Lead-Developer&Ja&Ja (Fehler- und Zugangslogs)\\
    Systemadministrator und DevOps&Ja&Ja (Alle Logs)\\
    \bottomrule
    \end{tabular}
\end{table}

\section{Kodierung der Ergebnisse}
Zur Kodierung der Transkriptionen aus den Befragungen wurden die Methoden nach Döring und Bortz gewählt: Den Textstellen (entspricht einem Teil des qualitativen Datenmaterials) aus den Ergebnissen wurden hierbei 
Codes zugewiesen, welcher eine Zusammenfassung bzw. eine Erklärung dieser darstellt \cite{Doring:2014aa} (vgl. Tabelle \ref{tab:coding}) - dieses Vorgehen steht die Vorgehensweise der deskriptiven Kodierung dar, mit
dessen Hilfe ähnliche Aussagen zusammengefasst werden können. Zum Schluss werden diese Codes noch ihren entsprechen, zugehörigen Kategorien zugeteilt, um eine allgemeine Abgrenzung gewährleisten zu können. 
\begin{table}
    \caption{Repräsentatives Codierungs- und Kategorisierungsbeispiel der erhobenen Daten (in Anlehnung an Döring und Bortz \cite{Doring:2014aa})} \label{tab:coding}
    \footnotesize
    \centering
    \begin{tabular}{r r r}
    \toprule
    & \multicolumn{2}{c}{\tabhead{Erhobene Daten}} \\ \cmidrule(lr){2-3} \tabhead{Textstelle} & \tabhead{Code} & \tabhead{EntsprechendeKategorie} \\ \midrule
    \enquote{Mich interessiert nicht, wer den Build zerstört hat.}&Desinteresse&Personenbezogene Daten irrelevant\\
    \enquote{Mir ist es wichtig, dass mein Team nicht nur im Chat rumhängt.}&Nutzeraktivitäten&Aktivitäten des Teams durch DevOps nachvollzogen\\
    \enquote{Ich nutze Slack nur wegen seiner Emojis.}&Desinteresse&Personenbezogene Daten irrelevant\\
    \bottomrule
    \end{tabular}
\end{table}