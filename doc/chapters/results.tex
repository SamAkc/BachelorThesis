% !TeX program = lualatex
\chapter{Ergebnisse} % Main chapter title
\label{Results} % For referencing the chapter elsewhere, use \ref{Results}
Anhand der Expertenbefragungen konnten insgesamt 13 verschiedene Kategorien zu den jeweiligen Sichtweisen der einzelnen Stakeholder herausausgearbeitet werden. Hierbei wird zwischen
der Präferenz der Privatsphäre im Allgemeinen oder der Benutzerfreundlichkeit bzw. Einfachheit unterschieden: Kategorien, in welchen Experten überwiegend
(>50\% der Experten sind der selben Ansicht) angegeben haben, die Privatsphäre sei ihnen wichtiger, werdem dem Kapitel \ref{privacy} zugeordnet, wohingegen die Kategorien, in welchen die Privatsphäre
aufgrund der Benutzerfreundlichkeit und Einfachheit vernachlässigt werden konnte, dem Kapitel \ref{noprivacy} zugerechnet werden. Im Anschluss werden die Ergebnisse,
in welchen kein einstimmiges Ergebnis (bei sieben Befragten ergibt dies: Weniger als 28\% der Befragten haben die selbe Meinung) erzielt werden konnte, in Kapitel \ref{noclearresult} thematisiert.

\section{Kategorien mit einstimmigen Ergebnissen}
\subsection{Kategorien mit einer Präferenz für die Benutzerfreundlichkeit und Einfachheit} \label{noprivacy}
\textbf{Verarbeitete Daten zur Unterstützung der Softwareentwicklung in DevOps-Tools:} \newline
Zu Beginn der Interviews haben alle Befragten angegeben, mit persönlichen Daten in Berührung zu kommen und diese auch zu verarbeiten. Dabei haben alle vier Softwareentwickler angegeben, die Nutzeraktivitäten und den Programmcode von Kollegen 
einsehen und verarbeiten zu können. Diese genannten Daten beinhalten Benutzernamen, Zeitstempel (in Git Commits, Builds und Deployments, z.T.)
Dies erfolge zur Nachvollziehbarkeit der Builds und Deployments, des Programmcodes selbst, z.B. durch Git Commits und zur allgemeinen Fehlerbehebung bei fehlgeschlagenen Builds bzw. fehlerhaftem Programmcode. 
Zwei Befragte haben zudem angegeben, Zugriff auf sensible Kundendaten in Form von Klarnamen, Anschriften, E-Mail Adressen uvm. zu besitzen: Dies ist bei jenen der Fall, die häufig mit externen Ansprechpartnern und Unternehmen in Kontakt treten 
und mit diesen zusammenarbeiten müssen. \newline \newline
\textbf{Bevorzugter Detailgrad von Zeitstempeln in DevOps-Tools:} \newline

\textbf{Allgemeine Auswertungen von persönlichen Daten aus DevOps-Tools:} \newline
Bei dieser Frage sollten die Befragten angeben, ob diese sich Auswertungsmöglichkeiten von persönlichen Daten aus DevOps-Tools vorstellen könnten, welche sie selbst nicht zwangsläufig durchführen, aber von welchen sie sich vorstellen könnten, dass dies bei anderen Kollegen bzw. Unternehmen
zum Einsatz kommen könnte. Hier haben alle Stakeholder angegeben, dass mit Sicherheit eine Form einer Leistungsbeurteilung bzw. -bewertung stattfinden könnte. Da in der Regel Benutzernamen und Zeitstempel in Builds, (Git) Commits und gelegentlich auch in Log-Dateien vorzufinden sind, könne man
anhand dieser Daten Rückschlüsse auf die Quantität und Qualität der einzelnen Faktoren ziehen. Ein Experte hat zudem angegeben, dass in seinem eigenen Unternehmen diese Daten häufig zum Einsatz zur Problembehebung kommen: Dabei wird anhand des Benutzernamens und Zeitstempels die Person hinter einem Build,
einem Git Commit oder einer anderen Ausführung ermittelt und durch gemeinsame Absprache mit diesem, eine Behebung des Problems angesteuert. Ein anderer Experte hat zusätzlich erläutert, Statistiken zum Vergleichen dieser Auswertungen mit persönlichen Daten erstellen zu können, da diese Daten unternehmensweit
für nahezu jeden Mitarbeiter einsehbar sind. Durch diese genannten Ansichten der verschiedenen Stakeholder zeigt sich eine unternehmensübergreifende Präferenz zur Benutzerfreundlichkeit bzw. Einfachheit in Bezug auf die Auswertungen der persönlichen Daten in DevOps-Tools. \newline \newline
\textbf{Interesse an persönlichen Daten in Review-Tools:} \newline

\textbf{Interesse an persönlichen Daten in Administrationstools:} \newline

\textbf{Detailgrad von Issues in Ticketverwaltungstools:} \newline

\textbf{Interesse an Benutzernamen und Zeitstempeln in Issue-Kommentaren in Ticketverwaltungstools:} \newline

\textbf{Tatsächliche Aufbewahrungsdauer von Log-Dateien:} \newline
Da ein sehr großer Anteil der Befragten von über 85\% (vgl. Tabelle \ref{tab:generaldata}) angegeben haben, Log-Dateien aktiv zu nutzen, ist es für den weiteren Verlauf der Analyse erforderlich, die tatsächliche Aufbewahrungsdauer der genutzten Log-Dateien 
im Unternehmen zu spezifizieren. Hier haben alle Befragten angegeben, dass alle Log-Dateien in der Regel für eine unbegrenzte bzw. unbestimmte Dauer aufbewahrt werden. Die Stakeholder haben zudem meist keine Kenntnis über die genaue Aufbewahrungsdauer und 
beziehen sich bei ihrer Aussage auf die Existenz von sehr alten Log-Dateien (über 5 Jahre). Die Ausnahme dieser Regel betrifft manuell durchgeführte Bereinigungen von Speicher und eingeleitete Löschung spezifischer Log-Dateien: Bei einem Befragten kam es vor,
dass das System einen Absturz erlitten hat, welche die Log-Dateien unbenutzbar gemacht haben, wohingegen bei einem anderen der Serverspeicher geleert werden musste und dieses Vorgehen sämtliche Log-Dateien gelöscht hat. \newline Aus diesen Aussagen lässt sich 
ableiten, dass ein Vorgehen zur automatisierten Löschung von Log-Dateien bei den Befragten bisher keine Anwendung gefunden hat. \newline \newline

\textbf{Mögliche Verarbeitungsinteressen von persönlichen Daten in anderen Unternehmen:} \newline

\subsection{Kategorien mit einer Präferenz für die Privatsphäre} \label{privacy}
\textbf{Interesse an persönlichen Daten in Log-Dateien:} \newline
Dieser Punkt befasst sich mit dem allgemeinen Interesse an persönlichen Daten, welche in Form von Benutzernamen, Zeitstempeln, IDs etc. in Log-Dateien sämtlicher Art auftreten können. Hier haben die Befragten einstimmig
entschieden, kein Interesse an diesen zu haben und ein mögliches Entfernen in der Zukunft willkommen zu heißen. Es wurde lediglich angegeben, dass in bestimmten Fällen (Zusammenarbeit an einem Projekt, 
in Testfällen von Code, Servern oder Applikationen oder auf Anfrage) eine Nachverfolgung auf Wunsch eines Kunden oder eines Auftrags erfolgen muss. Für die Befragten ist es nur wichtig, den Ablauf und auftretende Fehler 
von Servern, Applikationen oder Code im Allgemeinen nachverfolgen zu können - ein Interesse an bestimmten Personen oder Zeitstempeln ist nicht vorhanden, weswegen in dieser Kategorie die Privatsphäre über der Nutzerfreundlichkeit
steht. \newline \newline
\textbf{Bevorzugte Aufbewahrungsdauer von Log-Dateien:} \newline
Als Weiterleitung der Frage über die tatsächliche Aufbewahrungsdauer von Log-Dateien in Kapitel \ref{noprivacy} wurden in dieser Kategorie die Experten dazu angehalten, ihre persönliche Präferenz zur Dauer anzugeben. Hier haben alle 
Befragten angegeben, eine temporäre Aufbewahrungsdauer zu bevorzugen. Diese könne anhand der Lebensdauer von Projekten, der Lebensdauer von Programmcode, den letzten x Builds, einer festgelegten Zeit (z.B. zwei Wochen)
oder der Existenz von gemeldeten Fehlern deklariert werden. In der Regel sind sich die Befragten einig, eine Dauer von maximal wenigen Monaten zu bevorzugen. Diese Sichtweisen stellen einen Gegensatz zur tatsächlichen Aufbewahrungsdauer von
Log-Dateien in Softwareentwicklungsunternehmen dar, welche in Kapitel \ref{noprivacy} angesprochen werden.

\section{Kategorien mit unstimmigen Ergebnissen} \label{noclearresult}
\textbf{Detailgrad von Zeitstempeln in DevOps-Tools:} \\
\textbf{Sonstige Verarbeitungsinteressen von persönlichen Daten:} \\

\section{Vorgeschlagene Alternativen}

\section{Zusammenstellung der Ergebnisse}




