% !TeX program = lualatex
\chapter{Die Relevanz der Privatsphäre im 21. Jahrhundert - eine Einleitung} % Main chapter title
\label{Introduction} % For referencing the chapter elsewhere, use \ref{Introduction}

\enquote{We believe privacy is a fundamental human right} \cite{Apple:2020aa} - mit dieser Aussage betritt Craig Federighi, der Vizepräsident der Softwareentwicklung von Apple, einem Multimilliarden-Unternehmen der Software- und Hardwareentwicklung aus den Vereinigten Staaten, zur Vorstellung der 
neuen Privatsphäre-Richtliniendes Unternehmens zur sogenannten, firmeneigenen \enquote{Worldwide Developer Conference} die Bühne. \\ Und dieser Glaube ist nicht unbegründet: Mit der zunehmenden Digitalisierung in den letzten Jahrzehnten werden Menschen, und Internetnutzer im Spezifischen, 
häufiger vor dieses Dilemma gestellt. Entwicklungen wie dem Internet der Dinge (IoT), Industrie 4.0, \enquote{Smart Home}, Wearables (diverse Sensoren an der Kleidung selbst, Smartwatches, Fitnesstracker) uvm. ermöglichen neue (und zum Teil unerforschte) Angriffsmöglichkeiten für Cyberkriminalität \cite{Bundeskriminalamt:2018aa}.
Dies zeichnet sich vor allem durch den Anstieg der Höhe der Fallzahlen und der gleichzeitigen Abnahme der Entdeckungsquote der Malware aus: Im ersten Halbjahr 2018 wurden im Durchschnitt 13.000 Malware-Samples am Tag neu entdeckt \cite{GDB:18}, während diese Zahl im zweiten Halbjahr 2017 noch bei knapp 23.000 lag \cite{GDB:17}. \\ Ähnlich sieht es hier bei
dem Anstieg neuer Schadprogramme aus: Der G Data Blog, ein deutsches Softwareunternehmen, welches mehrfach für seine Sicherheitslösungen ausgezeichnet wurde \cite{GD:2020aa}, beschreibt dabei, dass die Zahl der neuen Schadprogrammtypen seit 2007 einen Anstieg um das knapp 63-fache vermerken konnte \cite{GDB:17}. 
Es ist anzunehmen, dass dies sicherlich auch damit zusammenhängt, dass das Internet in den letzten beiden Jahrzehnten einen regelrechten Ausbruch in der Nutzung erlebt hat. Während die Computer- und Internetnutzung 2012, einem Jahr, in welchem im Durchschnitt bereits knapp 46\% der Generation Plus50 Internetnutzer waren
\cite{GfK:2016aa}, noch für private Haushalte und Personen ab 10 Jahren bei 78\% lag \cite{Bundesamt:2019aa}, ist die Internetnutzung 2019 bis auf 88\% angestiegen \cite{Bundesamt:2019aa}. \\ \\
Nun könnte man anhand der steigenden Tendenz davon ausgehen, dass das Internet sich etabliert und die Nutzer durch den langjährigen Umgang und das gleichzeitige Sammeln von Erfahrung ein allgemeines Wissen über die Risiken und Gefahren im Bezug auf die eigene Sicherheit und Privatsphäre aufgebaut haben. Bei einer Umfrage durch 
Bitkom Research 2018 haben 89\% in der Altersgruppe ab 14 Jahren angegeben, sich in der Regel mit den Privatsphäre-Einstellungen auseinandergesetzt zu haben \cite{Bitkom:2018aa}. Auf der Kehrseite sieht man jedoch, wie beispielsweise diverse Bonusprogramme der DeutschlandCard und Payback GmbH, welche häufig in der Kritik von 
Verbraucherschützern stehen, da Nutzerdaten zu den Einkäufen und dem Einkaufsverhalten gesammelt werden \cite{Hatke:aa}, immer größer und verbreiteter werden: Alleine zum Vorjahr konnte die Payback GmbH im Jahre 2018 einen Anstieg des Jahresumsatzes auf 330,49 Millionen Euro in Deutschland und Österreich verzeichnen, während 2012 der Umsatz noch bei 148,57 Millionen 
Euro lag \cite{Payback:2019aa}. \\ Dieses Phänomen, auf der einen Seite seine Privatsphäre zu wahren und schützen wollen, auf der anderen Seite jedoch seine persönlichen Daten freiwillig auf Sozialen Netzwerken, beim Einkaufen oder durch die eigene Art und Weise, ein internetfähiges Gerät zu bedienen, zu teilen, nennt man das \enquote{Privacy Paradox} (dt. 
\textit{Privatsphäre-Paradoxon}) \cite{Barnes:2006aa} und wird im weiteren Verlauf eine Rolle spielen.
