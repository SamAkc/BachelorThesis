% !TeX program = lualatex
\chapter{Die Relevanz der Privatsphäre im 21. Jahrhundert - eine Einleitung} % Main chapter title
\label{Introduction} % For referencing the chapter elsewhere, use \ref{Introduction}

\enquote{We believe privacy is a fundamental human right} \cite{WWDC:20} - mit dieser Aussage betritt Craig Federighi, der Vizepräsident der Softwareentwicklung von Apple, einem Multimilliarden-Unternehmen der Software- und Hardwareentwicklung aus den Vereinigten Staaten, zur Vorstellung der 
neuen Privatsphäre-Richtliniendes Unternehmens zur sogenannten, firmeneigenen \enquote{Worldwide Developer Conference} die Bühne. \\ Und dieser Glaube ist nicht unbegründet: Mit der zunehmenden Digitalisierung in den letzten Jahrzehnten werden Menschen, und Internetnutzer im Spezifischen, 
häufiger vor dieses Dilemma gestellt. Entwicklungen wie dem Internet der Dinge (IoT), Industrie 4.0, \enquote{Smart Home}, Wearables (diverse Sensoren an der Kleidung selbst, Smartwatches, Fitnesstracker) uvm. ermöglichen neue (und zum Teil unerforschte) Angriffsmöglichkeiten für Cyberkriminalität \cite{BLB:18}.
Dies zeichnet sich vor allem durch den Anstieg der Höhe der Fallzahlen und der gleichzeitigen Abnahme der Aufklärungsquote aus: Im ersten Halbjahr 2018 wurden im Durchschnitt 13.000 Malware-Samples am Tag neu entdeckt \cite{GDB:18}, während diese Zahl 2017 noch bei [...] lag. \\ Ähnlich sieht es hier bei
dem Anstieg neuer Schadprogramme aus: Der Blog von G Data, einem deutschen Softwareunternehmen, welcher mehrfach für seine Sicherheitslösungen ausgezeichnet wurde \cite{GD+1}, beschreibt dabei, dass die Zahl der neuen Schadprogrammtypen seit 2007 einen Anstieg um das knapp 63-fache vermerken konnte \cite{GDB:17}.