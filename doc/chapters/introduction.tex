% !TeX program = lualatex
\chapter{Die Relevanz der Privatsphäre im 21. Jahrhundert - eine Einleitung} % Main chapter title
\label{Introduction} % For referencing the chapter elsewhere, use \ref{Introduction}

\enquote{We believe privacy is a fundamental human right} \cite{Apple 2020: 55:30 - 55:35} - mit dieser Aussage betritt Craig Federighi, der Vizepräsident der Softwareentwicklung von Apple, einem Multimilliarden-Unternehmen der Software- und Hardwareentwicklung aus den Vereinigten Staaten, zur Vorstellung der 
neuen Privatsphäre-Richtliniendes Unternehmens zur sogenannten, firmeneigenen \enquote{Worldwide Developer Conference} die Bühne. \\ Und dieser Glaube ist nicht unbegründet: Mit der zunehmenden Digitalisierung in den letzten Jahrzehnten werden Menschen, und Internetnutzer im Spezifischen, 
häufiger vor dieses Dilemma gestellt. Entwicklungen wie dem Internet der Dinge (IoT), Industrie 4.0, \enquote{Smart Home}, Wearables (diverse Sensoren an der Kleidung selbst, Smartwatches, Fitnesstracker) uvm. ermöglichen neue (und zum Teil unerforschte) Angriffsmöglichkeiten für Cyberkriminalität \cite{Bundeslagebild 2018: S. 52}.
Dies zeichnet sich vor allem durch den Anstieg der Höhe der Fallzahlen und der gleichzeitigen Abnahme der Aufklärungsquote aus: [...], liegt im ersten Halbjahr 2018 die Anzahl der neu entdeckten Malware-Samples am Tag im Durchschnitt bei 13.000 Fällen \cite{G DATA-Blog 2018}.