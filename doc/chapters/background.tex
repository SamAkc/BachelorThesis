% !TeX program = lualatex
\chapter{Theoretischer Hintergrund} % Main chapter title
\label{Background} % For referencing the chapter elsewhere, use \ref{Background}
Um in den folgenden Kapiteln ein grundlegendes Verständnis der Begrifflichkeiten, Zusammenhänge und Konzepte gewährleisten zu können, ist es notwendig,
diese vorab zu definieren und einzugrenzen. Da sich diese Arbeit mit der Privatsphäre von Stakeholdern in einem Softwareentwicklungsunternehmen beschäftigt,
ist es zunächst erforderlich, die Grenzen der Privatsphäre klar zu definieren und anhand dieser, eine Messung zu schaffen. Des weiteren müssen die Stakeholder,
welche in dieser Analyse betrachtet werden, identifiziert und im weiteren Verlauf in Gruppen gegliedert werden. Im Anschluss werden DevOps definiert und anhand
gewählter Beispiele, eine Analyse dieser aufgebaut.

\section{Die Privatsphäre - eine Eingrenzung}
Der Begriff Privatsphäre ist im Wortschatz der deutschen Sprache tief verankert und findet im Alltag in vielerlei Hinsicht Gebrauch: Menschen decken Notebook- und Smartphonekameras
ab, protestieren gegen Überwachungskameras im öffentlichen Raum \cite{Stallwood:2013aa} oder möchten nicht, dass ihre persönlichen und privaten Daten auf sozialen Netzwerken ohne ihre Zustimmung
bzw. ohne ihre Kenntnis verbreitet werden \cite{Picchi:2018aa} - doch was genau bedeutet Privatsphäre? \\
Die Privatsphäre lässt sich definieren als einen eingegrenzten, nicht-öffentlichen Raum, in welchem ein Individuum bzw. Individuen nach eigenem Belieben, ohne äußere Einflüsse oder die Beobachtung durch
Unbeteiligte, zur freien Entfaltung der eigenen Person, handeln kann \cite*{Pettinger:2020aa}. Möchte man diese Definition nun auf die Softwareentwicklung widerspiegeln, so müssen die möglichen, 
vorherrschenden Komponenten von persönlichen Daten näher betrachtet werden: In Unternehmen, welche sich mit der Softwareentwicklung befassen, müssen Daten vorhanden sein, welche Individuen individuell und 
bestimmen können (Grund liefern!) - dies kann in Form von IDs und individuell gewählten Nutzernamen vorzufinden sein, aber auf der Kehrseite auch mit Klarnamen, E-Mail Adressen sowie eindeutig identifizierbaren IDs. 

\section{Stakeholder}
Als Stakeholder werden jene \enquote{Personen, Gruppen oder Institutionen bezeichnet, die von den Aktivitäten eines Unternehmens 
direkt oder indirekt betroffen sind oder [...] ein Interesse an diesen [...] haben} \cite{Fleig:2016aa} - dies kann von Kunden 
und Lieferanten bis zu den eigenen Mitarbeitern und Eigentümern reichen \cite{Fleig:2016aa}. 

\section{DevOps}
\enquote{Die Zufriedenheit unserer Kunden liegt uns am Herzen} - durch dieses Motto strahlen Unternehmen ihre Kundenorientierung aus und möchten in der Regel zusätzlich durch Feedback, Kulanz, einem überzeugenden Produkt
oder einem allgemein positiv zurückgebliebenem Eindruck beim Kunden überzeugen. In Softwareentwicklungsunternehmen vereint DevOps die benötigten Technologien, die Prozesse und die Menschen miteinander, um für diese Kunden 
andauernd hochwertige Produkte anbieten zu können \cite{MSAzure:2020aa}. Der Begriff setzt sich zusammen aus \enquote{Development} (en. \textit{Entwicklung}) von Software sowie \enquote{Operations} (en. \textit{Vorgänge}), welche damit zusammenhängen 
\cite{MSAzure:2020aa} und repräsentiert dabei die \enquote{Zusammenarbeit von Entwicklung und Betrieb} \cite{Hasselbring:2015aa} dar. Dies erfolgt durch die Einführung von DevOps-Methoden und -Tools, sodass den zuvor getrennten Bereichen, 
und zusätzlichen wie beispielsweise der Qualitätssicherung, der Sicherheit etc., die Möglichkeit zur Koordination und Zusammenarbeit geboten wird \cite{MSAzure:2020aa}. \\ \\
Im Folgenden werden beispielhaft repräsentative DevOps-Methoden und -Tools aufgelistet und definiert.

\subsection{Continuous Integration und Delivery}

\subsection{Continuous Monitoring}

\subsection{Versionsverwaltung}

\subsection{DevOps-Tools in der Anwendung}