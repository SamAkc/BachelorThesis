% !TeX program = lualatex
\chapter{Theoretischer Hintergrund} % Main chapter title
\label{Background} % For referencing the chapter elsewhere, use \ref{Background}

\section{Definition und Identifikation der Stakeholder in einem Softwareentwicklungsunternehmen}
Ziel dieser qualitativen Analyse ist es, die Denk- bzw. Sichtweisen von Stakeholdern in Softwareentwicklungsunternehmen zu erfassen und anhand dieser Ergebnisse, Schlüsse in Bezug auf die
eigene Privatsphäre dieser zu ziehen. Um ein grundlegendes Verständnis in den folgenden Kapiteln gewährleisten zu können ist es zunächst notwendig, diese Stakeholder zu definieren und anhand davon, 
potenzielle Stakeholder im Rahmen dieser Forschung zu identifizieren. \\
In den folgenden Unterkapiteln werden die untergliederten Interessensgruppen und ihre entsprechende Notwendigkeit näher erläutert. Die Begründung für die spezifische Wahl dieser besagten Gruppen wird im
nächsten Kapitel klassifiziert.


\subsection{Berufseinsteiger, Praktikanten und Werkstudenten}

\subsection{Softwareentwickler}

\subsection{Team-Leads, Projektleiter (und Geschäftsführer?)}

\subsection{Externe Ansprechpartner}

\section{DevOps und deren Einsatz in Softwareentwicklungsunternehmen}

\subsection{Interne Kommunikationstools}

\subsection{Projektmanagement- und Wiki-Software}

\subsection{Git}

\section{Aktueller Stand und Technologien?}