% !TeX program = lualatex
\chapter{Theoretischer Hintergrund} % Main chapter title
\label{Background} % For referencing the chapter elsewhere, use \ref{Background}

\section{Die Privatsphäre - eine Eingrenzung}

\section{DevOps}

<<<<<<< HEAD
\subsection{Berufseinsteiger, Praktikanten und Werkstudenten}
Die primäre Gruppe zeichnet sich durch Softwareentwickler aus, welche kaum bis wenig Berufserfahrung besitzen und eventuell geringere Qualifikationen, als die anderen Interessensgruppen aufweisen könnten.
Diese Merkmale finden sich in Berufseinsteigern wieder, welche vor Kurzem aus einer abgeschlossenen Ausbildung oder eines abgeschlossenen Studiums stammen und nun im Beruf Softwareentwickler sind. Selbe Punkte
lassen sich allerdings auch auf (längerfristige) Praktikanten und Werkstudenten, welche neben ihrem Studium in einem Unternehmen bis zu 20 Stunden in der Woche tätig sind, übertragen - es ist also wenig Berufserfahrung
vorhanden, aber grundlegende Kenntnisse in der Informatik, vor allem in Bezug auf DevOps, welche im nächsten Unterkapitel näher erläutert werden, lassen sich in diesen Interessensgruppen wiederfinden.

\subsection{Softwareentwickler}
In der sekundären Gruppe befinden sich Softwareentwickler, welche als Festangestellte mehrjährige Erfahrungen in der Entwicklung in einem Unternehmen gesammelt haben. Diese sind in den meisten Fällen Teil eines Teams, welche gemeinsam an einem Projekt arbeiten
- im weiteren Verlauf der Analyse wird diese Personengruppe als \enquote{Softwareentwickler} genannt. Aufgrund der gesammelten Erfahrung hat diese Gruppe in der Regel auch häufig mit DevOps Kontakt und kann so begründetere und für die Analyse signifikantere
Aussagen als die primäre Gruppe treffen. Die grundlegenden Kenntnisse in der Informatik reichen weiter durch regelmäßige Anwendung in der Praxis, weswegen auch fachspezifischere Fragen mit Hinblick auf die Erfahrungen gestellt werden können.

\subsection{Team-Leads, Projektleiter (und Geschäftsführer?)}

\subsection{Externe Ansprechpartner}

\section{DevOps und deren Einsatz in Softwareentwicklungsunternehmen}

\subsection{Interne Kommunikationstools}
=======
\subsection{Git}
>>>>>>> 9eb2f86... further changes

\subsection{Projektmanagementtools}

\subsection{Interne Kommunikationstools und Wiki-Software}

\section{Stakeholder}